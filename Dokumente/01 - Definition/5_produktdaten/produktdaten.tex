\documentclass[../pflichtenheft.tex]{subfiles}

\newcommand{\pd}[1]{\item[\hypertarget{d#1}{/D#1/}]}

\begin{document}

\section{Produktdaten}
	\subsection{Komponente 1: Sensor Bibliothek} % wird für die PDF-Outline verwendet, deswegen keine Glossary-Verweise möglich
		\begin{itemize}
			\pd{101} Die Aufzeichnung der \gls{sensordata} in den letzten X Minuten\footnotemark[1]
		\end{itemize}
	\subsection{Komponente 2: Schritterkennungs Bibliothek}
		\begin{itemize}
			\pd{201} Die geschätzte Schrittfrequenz\footnotemark[1]
			\pd{202W} Anzahl der Schritte in dieser Session\footnotemark[1]
		\end{itemize}
	\subsection{Komponente 3: Anwendung}
		\begin{itemize}
			\pd{301} Die App-Einstellungen
			\pd{302} Zu der lokalen \gls{audiolib} (\falink{320}) hinzugefügte \Gls{audiofile}en:
				\begin{itemize}
					\item Name
					\item Pfad des Speicherorts
					\item \Gls{bpm}-Wert
					\item Weitere \Gls{metadata} (Titel, Künstler, ...)
				\end{itemize}
			\pd{303} Name der zuletzt abgespielten \Gls{audiofile}
			\pd{304} Zuletzt verbundene \Gls{earable}s (Geräte-Name)
			\pd{305W} Zugangsdaten eines \gls{spotify}-Accounts
			\pd{306W} Eine ausgewählte \gls{spotify}-Playlist
			\pd{307W} Anzahl der Schritte seit letztem Reset
		\end{itemize}

\footnotetext[1]{Diese Daten sollen nur temporär gespeichert werden.}
\let\thefootnote\relax\footnote{Mit W markierte PDs gehören zu den Wunschkriterien.}

\end{document}
