\documentclass[../pflichtenheft.tex]{subfiles}

\newcommand{\nf}[1]{\item[\hypertarget{nf#1}{/NF#1/}]}


\begin{document}

\section{Nichtfunktionale Anforderungen}
	\begin{itemize}
		\nf{010} Alle Komponenten sollen weitestmöglich plattformunspezifisch sein.
		\subsection{Komponente 1: Sensor Bibliothek}
			\nf{100} Die Sensor Bibliothek (\Gls{komp1}) soll ein hohes Maß an Konfigurierbarkeit bieten
			(etwa durch Spezifikation der gewünschten Sampling-Raten).
			\nf{110} Der Verbindungsaufbau (\falink{305}) zu den \Gls{earable}s soll in mindestens einem von zwei
			Fällen erfolgreich sein.
		\subsection{Komponente 2: Schritterkennungs Bibliothek}
			\nf{200} Die Schritterkennung (\falink{210}) muss bei einer Geschwindigkeit von 0,5 bis 3 Schritten pro
			Sekunde ordnungsgemäß funktionieren.
			\nf{210} Die Schritterkennung (\falink{210}) muss einen Schritt nach maximal 5 Sekunden registriert haben.
			\nf{220} Der Schrittzähler (\falink{221}) darf um maximal 30\% vom tatsächlichen Wert abweichen.
			\nf{230} Die ermittelte Schrittfrequenz (\falink{222}) darf um maximal 0,5 Schritte pro Sekunde vom
			tatsächlichen Wert abweichen.
		\subsection{Komponente 3: Anwendung}
			\nf{300} Der lokalen \Gls{audiolib} (\falink{320}) sollen mindestens 50 \Gls{audiofile}en hinzugefügt
			werden können.
			\nf{310} Das Hinzufügen von neuen \Gls{audiofile}en zur lokalem \Gls{audiolib} (\falink{321}) soll ohne
			Ermittlung des BPM-Wert maximal 2 Sekunden dauern.
			\nf{320} Das Finden eines zur Schrittfrequenz passenden Liedes (\falink{343}) soll maximal 5 Sekunden
			dauern.
			\nf{330} Der Wechsel zu den anderen \Gls{view}en soll mit maximal einer Sekunde Verzögerung ablaufen
			\nf{340} \Gls{user} begehen nach einstündiger Nutzung der \Gls{app} (\Gls{komp3}) nur noch maximal 2
			Fehler pro Tag.
			\nf{350W} Die automatisch ermittelte \Gls{bpm}-Wert einer \Gls{audiofile} (\falink{370W}) soll vom
			tatsächlichen Wert um maximal 15 \Gls{bpm} abweichen.
			\nf{360} Die Sprache soll ohne Neustart der App geändert werden können
			\nf{370} In einer Session des Motivationsmusik-Modus soll ein Lied nicht mehrmals hintereinander abgespielt werden
	\end{itemize}

	\let\thefootnote\relax\footnote{Mit W markierte NFs gehören zu den Wunschkriterien.}

\end{document}
