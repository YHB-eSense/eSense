\documentclass[../pflichtenheft.tex]{subfiles}

\newcommand{\gt}[1]{\item[\hypertarget{t#1}{/T#1/}]}

\begin{document}
	\section{Globale Testfälle}
		\subsection{Sensor Bibliothek: Testfälle}
			\begin{itemize}
				\gt{101} Geräte-Name der \Gls{earable}s auslesen (\falink{120})
				\gt{102} \Gls{batterystate} der \Gls{earable}s auslesen (\falink{130}, \falink{351})
				\gt{103} \Gls{sensor}en in Zeitraum I auslesen (\pdlink{101}, \falink{104})
				\gt{104} Ausgehenden \Gls{audiostream} öffnen (\falink{111})
				\gt{105} Ausgehenden \Gls{audiostream} schließen (\falink{112})
			\end{itemize}
		\subsection{Schritterkennungs Bibliothek: Testfälle}
			\begin{itemize}
				\gt{201} Schrittfrequenz berechnen (\falink{222})
				\gt{202} Schritte zählen (\falink{221})
				\gt{203} Langsame Schritte erkennen (0.5 Schritte/sec) (\falink{210})
				\gt{204} Schnelle Schritte erkennen (3 Schritte/sec) (\falink{210})
				\gt{205} Unreguläre Schrittmuster erkennen (\falink{210})
			\end{itemize}
		\subsection{Anwendung: Testfälle zum Verbindungsaufbau}
			\begin{itemize}
				\gt{301} Nach verfügbaren \Gls{bt}-Geräten suchen (\falink{302})
				\gt{302} Suche nach verfügbaren \Gls{bt}-Geräten aktualisieren (\falink{304})
				\gt{303} Verfügbare \Gls{bt}-Geräte auflisten (\falink{303})
				\gt{304} \Gls{ble}- und \Gls{bc}-Verbindung zu den \Gls{earable}s herstellen (\falink{305})
				\gt{305} \Gls{ble}- und \Gls{bc}-Verbindung zu den \Gls{earable}s trennen (\falink{306})
			\end{itemize}
		\subsection{Anwendung: Testfälle zur lokalen Audiobibliothek}
			\begin{itemize}
				\gt{401} \Gls{audiolib} öffnen (\falink{320})
				\gt{402} \Gls{audiofile} der \Gls{audiolib} hinzufügen (\falink{321}, \falink{322})
				\gt{403} \Gls{metadata} einer \Gls{audiofile} auslesen (\falink{323})
				\gt{404} \Gls{metadata} einer \Gls{audiofile} manuell bearbeiten (\falink{324})
				\gt{405} \Gls{audiofile} aus der \Gls{audiolib} löschen (\falink{325})
				\gt{406} \Gls{audiofile} auswählen und abspielen über \Gls{earable}s-Output (\falink{330}, \falink{331}, \falink{332})
				\gt{407} Wiedergabe pausieren bzw. wiedergeben (\falink{333})
				\gt{408} Nächste \Gls{audiofile} in \Gls{audiolib} abspielen (\falink{334})
				\gt{409} Vorherige \Gls{audiofile} in \Gls{audiolib} abspielen (\falink{334})
				\gt{411} \Gls{audiolib} durchsuchen nach Titel (\falink{326})
				\gt{412} \Gls{audiolib} durchsuchen nach Künstler (\falink{326})
				%\gt{413} \Gls{audiolib} durchsuchen nach \Gls{bpm}-Wert (\falink{326})
				\gt{414} \Gls{audiolib} nach Titel sortieren (\falink{327})
				\gt{415} \Gls{audiolib} nach Künstler sortieren (\falink{327})
				\gt{416} \Gls{audiolib} nach \Gls{bpm}-Wert sortieren (\falink{327})
				\gt{417} \Gls{audiofile} in \Gls{audiolib} finden, deren \Gls{bpm}-Wert am ehesten mit einem gegeben Wert übereinstimmt (\falink{343})
				\gt{418} \Gls{audiolib} schließen (\falink{320})
				\gt{419} Mehr als 50 Audiodateien zu der \Gls{audiolib} hinzufügen
			\end{itemize}
		\subsection{Anwendung: Testfälle zu den Einstellungen}
			\begin{itemize}
				\gt{501} Einstellungen öffnen (\falink{310})
				\gt{502} Sprache ändern (\falink{311})
				\gt{503} Schrittanzahl zurücksetzen (\falink{354})
				\gt{504} Einstellungen schließen (\falink{310})
			\end{itemize}
		\subsection{Anwendung: Testfälle zum Modus-Manager}
			\begin{itemize}
				\gt{601} Modus-Manager öffnen (\falink{340})
				\gt{602} Motivationsmusik-Modus anschalten (\falink{341})
				\gt{603} Motivationsmusik-Modus ausschalten (\falink{341})
				\gt{604} Aus Schrittfrequenz passenden \Gls{bpm}-Wert berechnen (\falink{342})
				\gt{605} Modus-Manager schließen (\falink{340})
			\end{itemize}
		\subsection{Anwendung: Testfälle zur Stabilität}
			\begin{itemize}
				\gt{701} \Gls{app} (\Gls{komp3}) öffnen
				\gt{702} \Gls{app} schließen
				\gt{703} Bildschirm drehen (Layout soll sich dabei nicht ändern)
				\gt{704} Bildschirm an- und ausschalten
				\gt{705} \Gls{app} im Hintergrund ausführen
			\end{itemize}
		\subsection{Anwendung: Testfälle zu den Wunschkriterien}
			\begin{itemize}
				\gt{801} \Gls{bpm}-Wert einer \Gls{audiofile} berechnen und gespeicherten \Gls{bpm}-Wert überschreiben
				\gt{802} Autostop-Modus anschalten (\falink{344W})
				\gt{803} Autostop-Modus ausschalten (\falink{344W})
				\gt{804} \Gls{app} (\Gls{komp3}) mit \Gls{spotify}-Account verbinden
				\gt{805} Modi des Modus-Managers in Kombination mit der persönlichen \Gls{spotify} \Gls{playlist} anstelle der lokalen \Gls{audiolib} nutzen
				\gt{806} Geräte-Name der \Gls{earable}s ändern (\falink{312W})
				\gt{807} Schneller Wechsel zwischen Stillstand und Bewegung
				\gt{808} Im Audio-Player die Lautstärke verändern (\falink{336W})
				\gt{809} Im Audio-Player den Abspielzeitpunkt verändern (\falink{337W})
			\end{itemize}
		\subsection{Anwendung: Testfälle zur Konfliktvermeidung}
			\begin{itemize}
				\gt{901} Solange die \Gls{earable}s nicht verbunden sind, sollen keine Abfragen an diese gestellt werden können (\falink{199})
				\gt{902} Solange keine \Gls{audiofile} geladen wurde, sollen die Funktionen \hyperlink{fa330}{\/FA33*\/} deaktiviert sein
				\gt{903} Es muss mindestens ein Lied in der \Gls{audiolib} vorhanden sein, bevor diese automatisch bezüglich eines \Gls{bpm}-Wertes durchsucht werden kann
			\end{itemize}

\newpage

\end{document}
