\documentclass[../pflichtenheft.tex]{subfiles}

%Kommando zum schnellen hinzufügen von Items
\newcommand{\fa}[1]{\item[\hypertarget{fa#1}{/FA#1/}]}
%Kommando zum Verlinken von Views
\newcommand{\view}[1]{\hyperlink{v#1}{/V#1/}}

\begin{document}

\section{Funktionale Anforderungen}

\subsection{Komponente 1: Sensor Bibliothek}
	\begin{itemize}
		\fa{100} Schnittstelle zum Auslesen der \Gls{sensordata} verbundener \Gls{earable}s
		\begin{itemize}
			%\fa{101} Momentanen Messwert eines \Gls{sensor}s abrufen
			%\fa{102} Bereitstellung einer sinnvollen Schnittstelle welche Zugriff auf die Sensordaten der Earables ermöglicht
			\fa{103} Aufzeichnung der \Gls{sensordata} für eine festgelegte Zeitspanne
			\fa{104} Abrufen der aufgezeichneten \Gls{sensordata}
		\end{itemize}
		\fa{110} Senden von \Gls{audiodata} zu verbundenen \Gls{earable}s
		\begin{itemize}
			\fa{111} Öffnen eines ausgehenden \Gls{audiostream}s zu den \Gls{earable}s
			\fa{112} Schließen eines geöffneten \Gls{audiostream}s
		\end{itemize}
		\fa{120} Konfiguration verbundener \Gls{earable}s
			\begin{itemize}
				\fa{121} Auslesen der Konfiguration\footnote{Gerätename; Aufzeichnungsbereich und Filter der Bewegungssensoren}
				\fa{122W} Ändern der Konfiguration
			\end{itemize}
		\fa{130} Auslesen des \Gls{batterystate}s verbundener \Gls{earable}s
		\fa{199} \falink{103}, \falink{111}, \falink{121} und \falink{122W} - nur, während \Gls{earable}s verbunden sind
	\end{itemize}

\subsection{Komponente 2: Schritterkennungs Bibliothek}
	\begin{itemize}
		\fa{200} Bereitstellung einer Eingabeschnittstelle für \Gls{sensordata}, welche mit \Gls{komp1} kompatibel ist
		\fa{210} Schritterkennung durch algorithmische Analyse der \Gls{sensordata}
		\fa{220} Darauf basierende \glspl{vsensor}
		\begin{itemize}
			\fa{221} Ein Schrittzähler
			\fa{222} Maß der momentanen Schrittfrequenz
		\end{itemize}
		\fa{230} Sinnvolle Schnittstelle zum Auslesen der virtuellen \Gls{sensordata}
	\end{itemize}

\subsection{Komponente 3: Anwendung}
	\begin{itemize}
		\fa{300} Anzeigen der Start-\Gls{view} (\view{10}) mit einem \Gls{bt}-Logo
		\begin{itemize}
			\fa{301} \view{10}: Klicken auf das Logo startet \falink{302}
			\fa{302} Es wird nach \Gls{bt}-Geräten gesucht und wechseln zu \view{20}
			\fa{303} \view{20}: Bei aktueller Suche gefundene Geräte werden aufgelistet
			\fa{304} \view{20}: Der \Gls{user} kann \falink{302} erneut ausführen lassen (Ergebnis aktualisieren)
			\fa{305} \view{20}: Wählt man ein Gerät aus der Liste, wird eine \Gls{bc}- sowie eine \Gls{ble}-Verbindung zu dem entsprechenden Gerät aufgebaut - bei Erfolg wechsel zu \view{11}, sonst \falink{302} wiederholen
			\fa{306} \view{11}: Klicken auf das Logo trennt die bestehende Verbindung
			\fa{307} \view{10}: Wechsel zu \view{11}, sobald eine Verbindung besteht
			\fa{308} \view{11}: Wechsel zu \view{10}, sobald keine Verbindung mehr besteht
		\end{itemize}
		\fa{310} Anzeigen des Einstellungsmenüs
		\begin{itemize}
			\fa{311} Der \Gls{user} kann die Sprache ändern
			\fa{312W} Der \Gls{user} kann den Geräte-Namen der derzeitig verbunden \Gls{earable}s ändern
			\fa{313} Der \Gls{user} kann den Wert des Schrittzählers (\falink{221}) zurücksetzen
		\end{itemize}
		\fa{320} Anzeigen der lokalen \Gls{audiolib} (\view{60})
		\begin{itemize}
			\fa{321} Der \Gls{user} kann \Gls{audiofile}en zur lokalen \Gls{audiolib} hinzufügen
			\fa{322} Zum wählen der \Gls{audiofile} soll das Dateiverzeichnis des \Gls{device}es angezeigt werden
			\fa{323} Beim Hinzufügen wird versucht die \Gls{metadata} der \Gls{audiofile} auszulesen. Gelingt dies nicht, muss der \Gls{user} diese manuell eingeben
			\fa{324} Der \Gls{user} kann die gespeicherten \Gls{metadata} einer \Gls{audiofile} bearbeiten
			\fa{325} Der \Gls{user} kann \Gls{audiofile}en aus der lokalen \Gls{audiolib} entfernen
			\fa{326} Der \Gls{user} kann die lokale \Gls{audiolib} durchsuchen nach Titel, Künstler oder \Gls{bpm}-Wert
			\fa{327} Der \Gls{user} kann die lokale \Gls{audiolib} sortieren nach Titel, Künstler oder \Gls{bpm}-Wert
			\fa{328} \falink{326}, \falink{327} - nur wenn Titel in der lokale \Gls{audiolib} vorhanden sind
		\end{itemize}
		\fa{330} Anzeigen des Audio-Players (\view{50})
		\begin{itemize}
			\fa{331} Wählt der \Gls{user} ein Lied aus der lokalen \Gls{audiolib} \falink{320} aus, so öffnet sich der Audio-Player und \falink{332} wird gestartet
			\fa{332} Das ausgewähltes Lied wird abgespielt
			\fa{333} Der \Gls{user} kann die Wiedergabe pausieren und wieder fortsetzen
			\fa{334} Der \Gls{user} kann vor- bzw. zurückspringen zum nächsten bzw. vorherigen Lied in der lokalen \Gls{audiolib} (\falink{320})
			\fa{335} \falink{333}, \falink{334}, \falink{336W}, \falink{337W} - erst wenn eine Audiodatei geladen wurde
			\fa{336W} Die Lautstärke verändern
			\fa{337W} Den Abspielzeitpunkt einer Audiodatei verändern
		\end{itemize}
		\fa{340} Anzeigen des Modus-Managers (\view{30})
		\begin{itemize}
			\fa{341} Der \Gls{user} kann den Motivationsmusik-Modus anschalten/ausschalten (sukzessive Ausführung von \falink{222}, \falink{342}, \falink{343}, \falink{332}, wobei sich dieser Vorgang wiederholt sobald ein Lied beendet wurde, aber kein Lied wird mehrfach hintereinander abgespielt)
			\fa{342} Aus Schrittfrequenz (\falink{222}) einen passenden \Gls{bpm}-Wert berechnen
			\fa{343} \Gls{audiolib} nach einer \Gls{audiofile} durchsuchen, deren \Gls{bpm}-Wert am ehesten mit einem gegebenen Wert übereinstimmt
			\fa{344W} Der \Gls{user} kann den Autostop-Modus anschalten/ausschalten, (Wiedergabe pausieren bzw. fortsetzen, je nachdem ob Schrittfrequenz (\falink{222}) auf Null fällt oder wieder ansteigt)
		\end{itemize}
		\fa{350} Anzeigen von Messwerten
		\begin{itemize}
			\fa{351} Den \Gls{batterystate} der verbundenen \Gls{earable}s in der Benachrichtigungszeile des Smartphones anzeigen
			\fa{352W} Bei niedrigem \Gls{batterystate} periodisch einen kurzen Piepston abspielen
			\fa{353} Den Wert des Schrittzählers (\falink{221}) in \view{11} anzeigen
			\fa{354} Den Gerätenamen der verbundenen \Gls{earable}s in \view{11} anzeigen
		\end{itemize}
		\fa{360W} Einbindung der \Gls{spotify} \Gls{api}
		\begin{itemize}
			\fa{361W} \falink{330} soll mit Liedern aus einer Spotify-Playlist des \Gls{user}s funktionieren
			\fa{362W} Bei der Ausführung von \falink{340} sollen Lieder aus einer Spotify-Playlist des \Gls{user}s verwendet werden
		\end{itemize}
		\fa{370W} Automatisches Berechnen des \Gls{bpm}-Wertes einer \Gls{audiofile}
	\end{itemize}

\let\thefootnote\relax\footnote{Mit W markierte FAs gehören zu den Wunschkriterien.}

\end{document}
