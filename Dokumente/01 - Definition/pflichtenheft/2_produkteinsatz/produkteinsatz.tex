\documentclass[../pflichtenheft.tex]{subfiles}

\begin{document}

	\section{Produkteinsatz}

		\begin{itemize}
			\item Das \Gls{product} wird im Rahmen der Veranstaltung \quote{Praxis der Softwareentwicklung} als studentisches Projekt entwickelt.
			\item Nach Abschluss soll dieses unter einer Open Source Lizenz öffentlich verfügbar gemacht werden.
			\item Das \Gls{product} soll die \Gls{sensordata} von \gls{esense} Kopfhörern auslesen und damit eine Schritterkennung
			implementieren.
			\item Das \Gls{product} soll mithilfe der Schritterkennung ein oder mehrere Anwendungsfälle implementieren.
			\item \Gls{komp1} und \Gls{komp2} sollen auch unabhängig von \Gls{komp3} verfügbar sein.
		\end{itemize}

		\subsection{Anwendungsbereich}
			\Gls{komp3} soll basierend auf dem Schrittmuster motivierende Musik wählen und gegebenenfalls auf Veränderung des Schrittmusters reagieren.

		\subsection{Zielgruppen}
			\begin{itemize}
				\item Sportliche junge \Gls{user}, welche beim Joggen oder Rennen motivierende Musik hören wollen.
				\item \Gls{developer}, welche die \Gls{library}en einbinden oder zum Ideen sammeln nutzen wollen.
			\end{itemize}

		\subsection{Betriebsbedingungen}
			Während der Nutzung soll der \Gls{user} zeitweise Gehen oder Rennen.

\end{document}
