\newglossaryentry{earable}
{
	name=Earable,
	description={Überbegriff für ein im Ohr getragenes Computersysteme. Wird in diesem Dokument häufig als Synonym zu \quote{\gls{esense}} verwendet}
}

\newglossaryentry{esense}
{
	name=eSense,
	description={Im Projekt verwendete \Gls{earable}-Plattform, welche \Gls{bt} zur Kommunikation verwendet. \see{http://www.esense.io/}}
}

\newglossaryentry{user}
{
	name=Benutzer,
	description={Person, welche die \Gls{earable}s trägt, um deren Funktionen zu verwenden}
}

\newglossaryentry{developer}
{
	name=Softwareentwickler,
	description={Person, welche Software für die \Gls{earable}s schreibt}
}

\newglossaryentry{device}
{
	name=Endgerät,
	description={Dem \Gls{user} zur Verfügung stehendes (tragbares) Computersystem, welches mit den \Gls{earable}s kommunizieren kann}
}

\newglossaryentry{sensor}
{
	name=Sensor,
	description={Bezeichnet Elektronik, die den realen Messwert einer physikalischen Größe ermittelt. Im Fall der \quote{\gls{esense}}-Plattform sind dies ein Gyroskop, ein Accelerometer sowie ein Druck-Sensor (Knopf)}
}

\newglossaryentry{vsensor}
{
	name=virtueller Sensor,
	plural=virtuelle Sensoren,
	description={Bezeichnet eine Softwarekomponente, die den zunächst unbekannten Wert einer physikalischen Größe aus anderen Messwerten abschätzt}
}

\newglossaryentry{batterystate}
{
	name=Akkuladezustand,
	description={Messwert der Betriebsspannung}
}

\newglossaryentry{library}
{
	name=Softwarebibliothek,
	description={Externe Sammlung verschiedener Softwareteile, welche in andere Software eingebunden werden kann, um deren Funktionalität zu erweitern}
}

\newglossaryentry{api}
{
	name=API,
	description={Software-Schnittstelle, welche die Kommunikation mit externen Software ermöglicht um so die Funktionalität zu erweitern}
}

\newglossaryentry{app}
{
	name=Anwendung,
	description={Software, welche zur Verwendung durch den \Gls{user} bestimmt ist, und daher eine grafische Benutzeroberfläche besitzt}
}

\newglossaryentry{view}
{
	name=Ansicht,
	description={Legt eine Aufbaustruktur für die grafische Benutzeroberfläche fest. Nur eine Ansicht kann zeitgleich aktiv sein}
}

\newglossaryentry{komp1}
{
	name=Komponente 1,
	description={Eine \Gls{library}, welche zur allgemeinen Kommunikation mit den \Gls{earable}s dient}
}

\newglossaryentry{komp2}
{
	name=Komponente 2,
	description={Eine \Gls{library}, welche \Gls{sensordata} der \Gls{earable}s verarbeiten kann und daraus aussagekräftigere Daten über das Gehverhalten des \Gls{user}s gewinnt}
}

\newglossaryentry{komp3}
{
	name=Komponente 3,
	description={Eine \Gls{app}, welche auf dem \Gls{device} lauffähig ist}
}

\newglossaryentry{product}
{
	name=Produkt,
	description={Sammelbegriff für alle drei Komponenten}
}

\newglossaryentry{bt}
{
	name=Bluetooth,
	description={Ein drahtloses Kommunikationsprotokoll. \see{https://www.bluetooth.com/}}
}

\newglossaryentry{ble}
{
	name=Bluetooth Low Energy,
	description={Moderne \Gls{bt} Variante, die sich durch einen geringen Stromverbrauch auszeichnet und von den \Gls{earable}s insbesondere zum Transfer der \Gls{sensordata} verwendet wird}
}

\newglossaryentry{bc}
{
	name=Bluetooth Classic,
	description={Herkömmliches \Gls{bt} (im Gegensatz zu \Gls{ble}), welches sich durch eine hohe Übertragungsrate auszeichnet. Dementsprechend wird es von den \Gls{earable}s zum Transfer von \Gls{audiodata} verwendet}
}

\newglossaryentry{ram}
{
	name=RAM,
	description={Arbeitsspeicher}
}

\newglossaryentry{sensordata}
{
	name=Sensordaten,
	description={Von \Gls{sensor}en erzeugte Werte. \glspl{vsensor} erzeugen virtuelle Sensordaten}
}

\newglossaryentry{audiodata}
{
	name=Audiodaten,
	description={Akustische Daten, welche von den \Gls{earable}s aufgenommen wurden oder abgespielt werden können}
}

\newglossaryentry{metadata}
{
	name=Metadaten,
	description={Übergeordnete Daten, welche neben dem eigentlichen Inhalt in einer Datei gespeichert werden (beispielsweise der \Gls{bpm}-Wert)}
}

\newglossaryentry{stepdata}
{
	name=Schrittmuster,
	description={Zeitlich geordnete Informationen über die vom \Gls{user} getätigten Schritte}
}

\newglossaryentry{audiolib}
{
	name=Audiobibliothek,
	description={Sammlung verschiedener Lieder mit zugehörigen \Gls{audiodata}}
}

\newglossaryentry{audiofile}
{
	name=Audiodatei,
	description={Datei, welche ein Lied in Form von \Gls{audiodata} enthält, sowie zusätzliche \Gls{metadata}}
}

\newglossaryentry{audiostream}
{
	name=Audio-Stream,
	description={Kontinuierliche Übertragung von \Gls{audiodata}}
}

\newglossaryentry{bpm}
{
	name=BPM,
	description={Taktschläge pro Minute (in der Regel bei einem Lied)}
}

\newglossaryentry{ios}
{
	name=iOS,
	description={Auf einigen \Gls{device}en verfügbares Betriebssystem. \see{https://www.apple.com/de/ios/}}
}

\newglossaryentry{android}
{
	name=Android,
	description={Auf einigen \Gls{device}en verfügbares Betriebssystem. \see{https://www.android.com/}}
}

\newglossaryentry{xamarin}
{
	name=Xamarin.Forms,
	description={Ein Framework, welches bei der Entwicklung von Cross-Platform-Applikationen unterstützt.
	\see{https://docs.microsoft.com/de-de/xamarin/}}
}

\newglossaryentry{spotify}
{
	name=Spotify,
	description={Ein Musik-Streaming-Dienst. \see{https://www.spotify.com/}}
}
\newglossaryentry{playlist}
{
	name=Playlist,
	description={Eine Liste von Liedern}
}

\newglossaryentry{i18n}
{
	name=Mehrsprachigkeit,
	description={Unterstützung einer Sprachauswahl für die Benutzeroberfläche der \Gls{app}}
}
