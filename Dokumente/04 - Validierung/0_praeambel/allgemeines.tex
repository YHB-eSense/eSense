\documentclass[../validierung.tex]{subfiles}

\begin{document}

\clearpage

\section{Allgemeines zum Testverfahren}
Um die korrekte Funktionsweise der App \Gls{karl} und der zugehörigen Software-Libraries zu gewährleisten,
wurde eine Kombination verschiedener Testverfahren angewandt.

\subsection{Komponententests mit XUnit}
Für die Komponententests haben wird das Framework \Gls{xunit} verwendet und in \gls{visualstudio} ein Test-Projekt erstellt.
Dieses enthält für jede öffentliche Klasse der App (beziehungsweise der Libraries) jeweils eine Testklasse, welche die zugehörigen Tests (für alle öffentlichen Methoden und Attribute) zusammenfasst.
XUnit unterscheidet dabei zwischen sogenannten \Gls{fact}s (Tests für invariante Bedingungen) und \Glspl{theory} (Testsfälle, welche nur bei Verwendung von bestimmten Eingabekombinationen bestehen und für andere Fehlschlagen).

\subsection{Testfallüberdeckung}
Um mögliche beim Testen außer Acht gelassenen Code-Teile leichter identifizieren zu können, haben wir die Testfallüberdeckung mit \gls{dotcover} ermittelt.

\subsection{Integrationstest}
Um die Schnitstellen zwischen App und Libraries zu testen, haben wir zudem zwei \Gls{testapp}s erstellt.

\subsubsection{EarableLibraryTestApp}
Die \quote{EarableLibraryTestApp} testet die Schnittstelle zu den Earables direkt auf einem Android- oder iOS-Smartphone.
Durch umfangreiche Testfälle kann so sichergestellt werden, dass bei der Verwendung der Karl-App kein Fehler in der Bluetooth-Konnektivität auftritt.

\subsubsection{StepDetectionTestApp}
Die \quote{StepDetectionTestApp} (für Windows) testet den Algorithmus zur Schritterkennung mit voraufgezeichneten Sensordaten direkt auf dem Entwicklungscomputer.
Auf diese Weise können die Auswirkung von Änderungen am Algorithmus unmittelbar festgestellt werden, ohne die Schritterkennung aufwendig manuell testen zu müssen.

\subsection{Systemtest}
Die in den vorherigen Phasen festgelegten Testfallszenarien dienen als Systemtests.


\end{document}
