\documentclass[../implementierung.tex]{subfiles}

\begin{document}

\section{Nichttriviale Fehler und Probleme}
\subsection{EarableLibrary}
Die größte Herausforderung beim Testen der EarableLibrary waren die durch die Bluetooth-Schnittstelle gegebenen Einschränkungen.
Da das verwendete \quote{Bluetooth LE plugin for Xamarin} keine Unterstützung für Windows bietet, konnten die Tests nicht direkt auf dem Entwicklungscomputer ausgeführt werden.
Um diese Einschränkung zu umgehen, haben wir eine auf Android und iOS lauffähige \quote{EarableLibraryTestApp} entwickelt.

Im Zuge der Tests sind mehrere Einschränkungen zutage getreten, welche durch das \quote{Bluetooth LE plugin}, das Betriebssystem oder die Earables selber gegeben sind.
Diese konnten dementsprechend nicht oder nur teilweise behoben werden.
\begin{itemize}
\item Zu dicht aufeinanderfolgendes Trennen und Wiederherstellen der Bluetooth-Verbindung zu den Earables ist nicht möglich. Obwohl der Vorgang zunächst gelingt, geht wenige Sekunden später die Verbindung zu den Earables verloren und muss erneut hergestellt werden.
Das Einbauen einer zweisekündigen Verzögerung vor dem Wiederverbindungsvorgang war in allen Testfällen ausreichend, um ein Abbrechen der Verbindung zu vermeiden.
\item Der Name, welcher den ESense-Earables zugewiesen werden kann, ist auf 11 Zeichen beschränkt. Der Versuch, einen längeren Namen zuzuweisen wird vom ESense ignoriert und der alte Name wird beibehalten.
\end{document}
