\documentclass[../implementierung.tex]{subfiles}

\begin{document}

\section{Nichttriviale Fehler und Probleme}
\subsection{EarableLibrary}
Die größte Herausforderung beim Testen der EarableLibrary waren die durch die Bluetooth-Schnittstelle gegebenen Einschränkungen.
Da das verwendete \quote{Bluetooth LE plugin for Xamarin} keine Unterstützung für Windows bietet, konnten die Tests nicht direkt auf dem Entwicklungscomputer ausgeführt werden.
Um diese Einschränkung zu umgehen, haben wir eine auf Android und iOS lauffähige \quote{EarableLibraryTestApp} entwickelt.

Um dennoch Komponententests durchführen zu können war es zudem nötig, die Bluetooth-Verbindung zu simulieren.
Da die dazu notwendige Logik über viele Klassen verteilt war, hat es sich als sinnvoll erwiesen, sämtliche Verbindungs-Funktionalitäten in eine eigene Klasse
\quote{BLEConnection} zu verschieben und dadurch die Kopplung zwischen EarableLibrary und dem unterliegendem \quote{Bluetooth LE plugin} zu reduzieren.

Im Zuge der Tests sind mehrere Einschränkungen und Fehler aufgefallen, welche sich durch Änderungen an der EarableLibrary nicht beheben ließen.
Nähere Begutachtung zeigte, dass diese auf das \quote{Bluetooth LE plugin}, das Betriebssystem oder aber die Earables selbst zurückzuführen waren.
\begin{itemize}
\item Zu dicht aufeinanderfolgendes Trennen und Wiederherstellen der Bluetooth-Verbindung zu den Earables ist nicht möglich. Obwohl der Vorgang zunächst gelingt, geht wenige Sekunden später die Verbindung zu den Earables verloren.
Das Einbauen einer zweisekündigen Verzögerung vor dem Wiederverbindungsvorgang war in allen Testfällen ausreichend, um ein Abbrechen der Verbindung zu vermeiden.
Alternativ wäre eine automatische Wiederverbindung nach dem Verbindungsabbruch möglich.
\item Der Name, welcher den eSense-Earables zugewiesen werden kann, ist in seiner maximalen Länge auf 11 Zeichen beschränkt.
Namen, die länger als 20 Zeichen sind, werden vom Earable nicht übernommen.
Hat der Name hingegen eine Länge zwischen 12 und 20 Zeichen, so tritt ein interessantes Verhalten zutage:
Der Name wird vorerst in voller Länge übernommen und übersteht auch Neuverbindungen und Neustarts der Kopfhörer.
Nach einiger Zeit jedoch, oder spätestens wenn die Kopfhörer zurück in die Ladeschale gelegt werden, wird der Name automatisch auf 11 Zeichen gekürzt!
\end{itemize}

\end{document}
