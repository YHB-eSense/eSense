\documentclass[../validierung.tex]{subfiles}

\begin{document}


\section{Nichttriviale Fehler und Probleme}
	\subsection{EarableLibrary}
		Die größte Herausforderung beim Testen der EarableLibrary waren die durch die Bluetooth-Schnittstelle gegebenen Einschränkungen.
		Da das verwendete \quote{Bluetooth LE plugin for Xamarin} keine Unterstützung für Windows bietet, konnten die Tests nicht direkt auf dem
		Entwicklungscomputer ausgeführt werden.
		Um diese Einschränkung zu umgehen, haben wir eine auf Android und iOS lauffähige \quote{EarableLibraryTestApp} entwickelt.
		
		Um dennoch Komponententests durchführen zu können war es zudem nötig, die Bluetooth-Verbindung zu simulieren.
		Da die dazu notwendige Logik über viele Klassen verteilt war, hat es sich als sinnvoll erwiesen, sämtliche Verbindungs-Funktionalitäten in eine 
		eigene Klasse \quote{BLEConnection} zu verschieben und dadurch die Kopplung zwischen EarableLibrary und dem unterliegendem 
		\quote{Bluetooth LE plugin} zu reduzieren.
		
		Im Zuge der Tests sind mehrere Einschränkungen und Fehler aufgefallen, welche sich durch Änderungen an der EarableLibrary nicht beheben ließen.
		Nähere Begutachtung zeigte, dass diese auf das \quote{Bluetooth LE plugin}, das Betriebssystem oder aber die Earables selbst zurückzuführen waren.
		
		\begin{itemize}
			\item Zu dicht aufeinanderfolgendes Trennen und Wiederherstellen der Bluetooth-Verbindung zu den Earables ist nicht möglich. 
			Obwohl der Vorgang
			zunächst gelingt, geht wenige Sekunden später die Verbindung zu den Earables verloren. Das Einbauen einer zweisekündigen Verzögerung vor dem
			Wiederverbindungsvorgang war in allen Testfällen ausreichend, um ein Abbrechen der Verbindung zu vermeiden.
			Alternativ wäre eine automatische Wiederverbindung nach dem Verbindungsabbruch möglich.
			\item Der Name, welcher den eSense-Earables zugewiesen werden kann, ist in seiner maximalen Länge auf 11 Zeichen beschränkt. Namen, die länger
			als 20 Zeichen sind, werden vom Earable nicht übernommen. Hat der Name hingegen eine Länge zwischen 12 und 20 Zeichen, so tritt ein 
			interessantes Verhalten zutage: Der Name wird vorerst in voller Länge übernommen und übersteht auch Neuverbindungen und Neustarts der 
			Kopfhörer. Nach einiger Zeit jedoch, oder spätestens wenn die Kopfhörer zurück in die Ladeschale gelegt werden, wird der Name automatisch auf 
			11 Zeichen gekürzt! 		
		\end{itemize}

	\subsection{SpotifyAudioModule}
		Für die Integration von Spotify in die Anwendung wurde ein Plugin verwendet, welches ermöglicht die Funktionen der Spotify Web API zu verwenden. 
		Da allerdings nur auf Lieder eines Nutzers, die sich in einer Playlist befinden per Spotify Web API zugegriffen werden kann, wurden Playlists in 
		der Implementierungsphase zur AudioLib hinzugefügt. Beim Testen fiel jedoch auf, dass Nutzer ohne Playlists einfach mit einer leeren AudioLib 
		konfrontiert wären, was die Benutzerfreundlichkeit der Anwendung einschränken würde.\\ Um dieses Problem zu umgehen, wurde deshalb eine Meldung 
		integriert, welche dem Nutzer beim Ändern des Audiomodules in einem solchen Fall aufklärt. Anschließend wird das Audiomodul nun zum 
		BasicAudioModule zurückgesetzt.
	\subsection{MainPageVM}
		Um sich in der Anwendung über Bluetooth mit eine Earables-Paar zu verbinden, muss der Nutzer bekanntlich auf der MainPage der mittleren Knopf 
		betätigen, um anschließend zu den Systemeinstellungen des Geräts weitergeleitet zu werden, wo dieser sich mit den Earables verbinden kann. Hat der 
		Nutzer dies erledigt, muss dieser den Knopf erneut betätigen um die BLE-Verbindung zu den Earables herzustellen. Da der Nutzer bei wiederholter 
		Betätigung des Knopfs die BLE-Verbindung trennt, kann es leicht dazu kommen, dass dies aus Versehen geschieht.\\ Um dieses Problem zu vermeiden 
		und die Benutzerfreundlichkeit der Anwendung aufrecht zu erhalten, wurde eine Meldung integriert, die den Nutzer bei versuchter Trennung der 
		Verbindung warnt.
	\subsection{AudioLibPageVM}
		Beim Ändern des Audiomoduls in der Anwendung vom BasicAudioModule hin zum SpotifyAudioModule wurde eine NotImplementedException geworfen. Diese 	
		hatte den Grund, dass die AudioLibPage schon auf die Playlists und SelectedPlaylists Properties der AudioLibPageVM zugegriffen hat, während diesen 
		schon erlaubt war die entsprechenden Properties aus dem Model zurückzugeben, bevor das Audiomodul auf das SpotifyModule gesetzt wurde. Somit wurde 
		auf die Playlists und SelectedPlaylist Properties der BasicAudioLib zugegriffen, was der Grund für die Exception war, da die BasicAudioLib dies 
		nicht implementiert.\\ Um diesen Fehler zu beheben, wurden die Reihenfolge der sich verändernden Properties und aufgerufenen Events angepasst.
	\subsection{BasicAudioLib, AudioLib und AudioLibPageVM}
		Das Löschen von Audiodateien aus der BasicAudioTrackDatabase geschieht asynchron, aus offensichtlichen Gründen. Jedoch wurde die Methode 
		DeleteTrackAsync der BasicAudioTrackDatabase in der BasicAudioLib und so auch in der AudioLib und AudioLibPageVM synchron aufgerufen, sodass es 
		vereinzelt dazu kam, dass die AudioLib in der Anwendung beim Löschen von Dateien nicht richtig aktualisiert wurde.\\ Um diesen Fehler zu beheben, 
		wurden die Methoden der BasicAudioLib, AudioLib und AudioLibPageVM so angepasst, dass nun die Methode DeleteTrackAsync der BasicAudioTrackDatabase 
		asynchron aufgerufen wird.

	\subsection{Generelle Fehler}
		\begin{itemize}
			\item Es ist aufgefallen, dass wir deutlich zu wenig gegen interfaces implementiert haben.
			\item Auch deutlich zu oft wurden im Konstruktor neue Objekte erstellt, oft wäre es besser gewesen die Objekte als Parameter zu übergeben. (Dependency Injection)
		\end{itemize}
		$\rightarrow$ Beides zusammen führte dazu, dass das Projekt schwer zu testen war.

\end{document}
