\documentclass[../implementierung.tex]{subfiles}

\begin{document}

\section{Integrationstests}

Um die Libraries als Gesamteinheit zu testen, haben wir mehrere Integrationstests geschrieben. Diese Testen die verschiedenen Funktionalitäten der Libraries im Zusammenhang. Diese Tests waren oft schwerer zu realisieren, da äußere Datensätzen vonnöten sind und wir die Tests nicht ausschließlich auf dem PC durchführen kann.

\subsection{EarableLibraryTestApp}

	Die \quote{EarableLibraryTestApp} testet die Funktionsfähigkeit der EarableLibrary mit einem echten eSense-Kopfhörer.

	\subsubsection{Test-Ablauf}
		Grundvoraussetzung für die TestApp ist, dass ein eSense-Kopfhörer in Reichweite ist und bereits über die Systemeinstellungen mit dem Test-Smartphone gekoppelt wurde.
		Dies wird gleich zu Beginn überprüft um zu vermeiden, dass daraus resultierende Fehler im späteren Testverlauf falsch interpretiert werden.
		War die Verbindung zum Kopfhörer erfolgreich, beginnt die sequentielle Ausführung der folgenden Testfälle:
		\begin{itemize}
			\item \code{ConnectionTest} Wiederholtes (Wieder-)Herstellen oder Trennen der Verbindung zu den Earables soll stets in einem Wohldefinierten Zustand resultieren.
			\item \code{SensorTest<MotionSensor>} Die inertiale Messeinheit muss per \quote{Subscription}-Mechanismus gültige Messwerte zur Verfügung stellen.
			\item \code{SensorTest<PushButton>} Der Durckknopf muss mittels \quote{Subscription}-Mechanismus und bei direktem Auslesen gültige Messwerte zur Verfügung stellen.
			\item \code{SensorTest<VoltageSensor>} Der Spannungssensor muss bei direktem Auslesen gültige Messwerte zur Verfügung stellen.
			\item \code{NameChangeTest} Der Name des Kopfhörers muss ausgelesen und geändert werden können. Zudem müssen Änderungen über Wiederverbindungen hinaus persistent sein.
		\end{itemize}

	\subsubsection{Test-Resultat}
		Nach Beendigung des Gesamt-Tests werden die Resultate für alle Testfälle angezeigt (\quote{Passed} oder \quote{Failed}).
		Im Fehlerfall wird zudem die Fehlerursache mit angegeben.

\end{document}
