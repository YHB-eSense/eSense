\documentclass[../implementierung.tex]{subfiles}

\begin{document}

\subsection{StepDetectionTestApp}

	Die \quote{StepDetectionTestApp} führt den Schritterkennungsalgorithmus mit zuvor aufgezeichneten Daten aus.

	\subsubsection{Funktionsprinzip}
		Die von uns verwendeten Daten beinhalten aufgezeichnete Schritte, aufgezeichnetes Nichts-Tun und wildes Schütteln der Kopfhörer.
		Diese Daten werden dann an den Algorithmus geliefert, als kämen diese direkt vom Earable.

	\subsubsection{Algorithmus-Tuning}
		Unter Verwendung von geeigneten Grenzwerten ist es uns gelungen, eine Schritterkennungsrate von über 95\% der tatsächlichen Schritte zu erzielen.
		Nicht möglich war jedoch die vollständige Vermeidung von Falschpositiven.
		Mit dem von uns verwendeten Schwellwertalgorithmus können manche Bewegungen (wie etwas das Schütteln der Kopfhörer) nicht von Schritten unterschieden werden.
		Nur ein gänzlich anderer Algorithmus-Ansatz (wie etwa maschinelles Lernen) hätte hier Abhilfe leisten können.
		Dies war für den Einsatz in unserer App (als Schritterkennung beim Joggen) jedoch gar nicht nötig, da der Anteil der Falschpositiven beim Joggen anteilmäßig nicht ins Gewicht fällt.

\end{document}
