\documentclass[../implementierung.tex]{subfiles}
\begin{document}
\subsection{\# Interface Mode}
\begin{itemize}
\remove{  Wurde nicht benötigt. \sign{protected abstract String UpdateName(Lang value)}}
\end{itemize}
\subsection{\# AutostopMode}
Implementiert jetzt auch \sign{IObserver<Output>}, um Daten von der StepDetectionLibrary zu erhalten.
\begin{itemize}
\add{Wenn die Property gesetzt wird, wird das Playback des Users gestoppt/weiter abgespielt. \sign{public bool Autostopped}}
\end{itemize}
\subsection{\# MotivationMode}
Implementiert jetzt auch \sign{IObserver <Output>}, um Daten von der StepDetectionLibrary zu erhalten.
\begin{itemize}
\add{Wenn die aktuelle Schrittfrequenz und die BPM des laufenden Songs sich um mehr als den Schwellwert(also MaxAllowedBPMDiff) unterscheiden wird ein neuer (passender) Song abgespielt.\sign{public double MaxAllowedBPMDiff = 15} }
\add{ Wählt einen Song aus dessen BPM zur Schrittfrequenz des Users passt. \sign{public void ChooseNextSong()}}
\end{itemize}
\end{document}