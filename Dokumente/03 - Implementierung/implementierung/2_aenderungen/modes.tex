\documentclass[../implementierung.tex]{subfiles}
\begin{document}

\subsection{\# Interface Mode}
	\begin{itemize}
		\remove[protected abstract String UpdateName(Lang value)]{Wurde nicht benötigt.}
	\end{itemize}

\subsection{\# AutostopMode}
	Implementiert jetzt auch \code{IObserver<Output>}, um Daten von der StepDetectionLibrary zu erhalten.
	\begin{itemize}
		\add[public bool Autostopped]{Wenn die Property gesetzt wird, wird das Playback des Users gestoppt/weiter abgespielt.}
	\end{itemize}

\subsection{\# MotivationMode}
	Implementiert jetzt auch \sign{IObserver <Output>}, um Daten von der StepDetectionLibrary zu erhalten.
	\begin{itemize}
		\add[public double MaxAllowedBPMDiff = 15]{Wenn die aktuelle Schrittfrequenz und die BPM des laufenden Songs sich um mehr als den Schwellwert(also \code{MaxAllowedBPMDiff}) unterscheiden wird ein neuer (passender) Song abgespielt.}
		\add[public void ChooseNextSong()]{Wählt einen Song aus dessen BPM zur Schrittfrequenz des Users passt.}
	\end{itemize}

\end{document}
