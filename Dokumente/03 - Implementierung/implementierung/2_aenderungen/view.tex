\documentclass[../implementierung.tex]{subfiles}
\begin{document}

\section{View}
Aus allen View Klassen wurde die Methode \code{OnAppearing()} entfernt. Aktualisierungen finden nur nach
 Änderungen von Werten der UI Elemente statt.

\subsection{\# MainPage}
	\begin{itemize}
		\add[ImageButton \_helpButton]{macht durch Drücken die vier Hilfsicons sichtbar/unsichtbar.}
		\add{Vier Hilfsicons(Images) zeigen an, durch welche Geste man zu welcher Seite der App navigiert.}
	\end{itemize}

\subsection{- ConnectionPage}
	Diese Klasse wurde komplett entfernt. Der Nutzer wird bei Betätigung des ConnectionButtons der MainPage stattdessen zu den Settings des Smartphones geleitet um dort sich mit einem Earable über Bluetooth-Classic zu verbinden, sollte noch keine Verbindung bestehen. Andernfalls wird sich bei Betätigung des ConnectionButtons der MainPage direkt mit den in den Settings verbundenen Earables über Bluetooth-Low-Energy verbunden.

\subsection{\# AddSongsPage}
	\begin{itemize}
		\add[Button \_calculateBPMButton]{berechnet beim Drücken die BPM der durch den AudioFilePicker ausgewählten Audiodatei.}
		\add[Label \_songTitleLabel]{Label zum \_songTitleEntry.}
		\add[Label \_songArtistLabel]{Label zum \_songArtistEntry.}
		\add[Label \_songBPMLabel]{Label zum \_songBPMEntry.}
	\end{itemize}

\subsection{\# AudioLibPage}
	\begin{itemize}
		\add[Picker \_playlistPicker]{ermöglicht es, eine Playlist der SpotifyAudioLib auszuwählen, deren AudioTracks anschließend in die AudioLib geladen werden.}
		\add[ImageButton \_deleteSongButton] {Button zum Löschen eines Songs aus der BasicAudioLib.}
		\change[ImageButton \_addSongButton] {Typ wurde von Button auf ImageButton geändert.}
	\end{itemize}
\subsection{\# AudioPlayerPage}
	\begin{itemize}
		\add[Entry \_titleEntry]{ermöglicht es den Titel eines neu hinzuzufügenden Songs festzulegen.}
		\change[ImageButton \_nextSongButton, \_playPauseSongButton, \_previousSongButton]{Typ wurde von Button auf ImageButton geändert.}
	\end{itemize}

\subsection{\# ModesPage}
	\begin{itemize}
		\add[ChartView \_stepChart]{zeigt für die letzten 10 Minuten an, wie viele Schritte jede Minute gemacht wurden.}
	\end{itemize}

\subsection{\# SettingsPage}
	\begin{itemize}
		\add[Button \_changeAudioModuleButton]{ermöglicht das Wechseln zwischen der BasicAudioLib und der SpotifyAudioLib bzw. dem BasicAudioPlayer und dem SpotifyAudioPlayer.}
		\add[Picker \_colorPicker]{ermöglicht das Wechseln der Farbe vieler UI-Elemente der App.}
		\add[Label \_colorPickerLabel]{fungiert als Titel für den ColorPicker und passt sich an die momentan ausgewählte Sprache an.}
		\add[Label \_changeDeviceNameLabel]{Label zum \_changeDeviceNameEntry.}
	\end{itemize}

\end{document}
