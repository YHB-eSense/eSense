\documentclass[../implementierung.tex]{subfiles}
\begin{document}
\section{View}
Aus allen View Klassen wurde die Methode OnAppearing() entfernt. Aktualisierungen finden nur nach
 Änderungen von Werten der UI Elemente statt.
\subsection{\# MainPage} 
\begin{itemize}
\add{Ein Button hinzugefügt, der durch Drücken Hilfssymbole sichtbar/unsichtbar macht, die anzeigen durch welche Geste man zu welcher Seite der App navigiert}
\add{Vier Images (Hilfssymbole) , die anzeigen durch welche Geste man zu welcher Seite der App navigiert}
\end{itemize}
\subsection{- ConnectionPage} Wurde komplett gelöscht, da es bei iOS nicht möglich ist sich über eine App mit Bluetooth Geräten zu verbinden.
\subsection{\# AddSongsPage}
\begin{itemize}
\add{Ein Button, um die BPM des ausgewählten Songs zu berechnen}
\end{itemize}
\subsection{\# AudioLibPage}
\begin{itemize}
\add{Ein Picker für Playlist, die aus der Spotify Audio Lib angezeigt werden soll, wurde hinzugefügt}
\end{itemize}
\subsection{\# AudioPlayerPage} 
\begin{itemize}
\add{Ein Textfeld für den Titel eines Songs}
\end{itemize}
\subsection{\# ModesPage} 
\begin{itemize}
\add{Ein ChartView zum Anzeigen der Schritte in einem festgelegten Zeitraum }
\end{itemize}
\subsection{\# SettingsPage}
\begin{itemize}
\add{ Ein Button zum Wechseln zwischen der BasicAudioLib und der SpotifyAudioLib }
\add{ Es wurde ein Picker zum Wechseln der Farbe der UI Elemente der App und ein Label mit der Aufschrift "Farbe" ( 'Farbe' in der ausgewählten Sprache) hinzugefügt }
\end{itemize}
\end{document}
