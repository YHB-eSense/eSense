\documentclass[../implementierung.tex]{subfiles}
\begin{document}

\section{Earable Library}

\subsection{\# IEarableScanner \changeto{} IEarableManager}
	Eine wesentliche Änderung im Prinzip der Library hat sich beim Herstellen der erstmaligen Verbindung zu einem Earable ergeben.
	Ursprünglich sollte die EarableLibrary in der Lage sein, zu einem gefundenen Gerät eine BLE- und eine BC-Verbindung aufzubauen.
	Wie sich heraus stellte, besitzt die iOS-Plattform jedoch Einschränkungen, welche es unmöglich machen, programmgesteuert eine BC-Verbindung
	mit einem der gefundenen Geräte einzugehen.

	Um diese Einschränkung zu umgehen, muss der Benutzer selbstständig (über die Systemeinstellungen) eine BC-Verbindung mit
	den gewünschten Earables herstellen, bevor über die Earables Musik abgespielt werden kann.
	Da der Benutzer nun sowieso das gewünschte Gerät in den Systemeinstellungen auswählen muss, hat es sich als nutzerfreundlicher erwiesen,
	beim Verbinden seitens der EarableLibrary auf einen Scan zu verzichten und stattdessen das momentan über BC verbundene Gerät als Earable zu verbinden.

	Infolgedessen wurde der IEarableScanner als IEarableManager mit verändertem Funktionsumfang neu gestaltet.
	\begin{itemize}
		\remove[event EventHandler<EarableEventArgs> EarableDiscovered]{}
		\remove[void StartScanning()]{}
		\remove[void StopScanning()]{}
		\add[List<IEarable> ListEarables()]{}
		\add[Task<IEarable> ConnectEarableAsync()]{}
	\end{itemize}

\subsection{\# EarableLibrary}
	\sign{public class EarableLibrary : IEarableManager}
	Implementiert \code{IEarableManager}.

\subsection{EarableEventArgs}
	Keine Änderungen.

\subsection{\# IEarable}
	Der AudioStream fällt weg, da die Audio-Ausgabe nun vom Betriebssystem verwaltet wird.
	Des weiteren musste der Schreibzugriff auf den Gerätenamen in eine separate Methode verschoben werden, damit dieser asynchron erfolgen kann.
	Zuletzt wurde der Zugriff auf die Sensoren leicht abgeändert, um diesen bequemer zu gestalten.
	\begin{itemize}
		\remove[IAudioStream AudioStream]{}
		\change[string Name \{ set; \} \changeto{} void SetNameAsync(string)]{}
		\change[ReadOnlyCollection<ISensor> sensors \changeto{} T GetSensor<T>() where T : ISensor]{}
	\end{itemize}

\subsection{\# ESense}
	\sign{public class ESense : IEarable}
	Implementiert \code{IEarable}.
	Die folgenden Sensoren sind verfügbar: \code{MotionSensor}, \code{PushButton}, \code{VoltageSensor}
	\begin{itemize}
		\add[public static Guid\array{} ServiceUuids]{Enthält die eindeutigen IDs derjenigen BLE-Services,
			welche vom verbundenen BT-Gerät unterstützt werden müssen, um die uneingeschränkte Funktionalität
			der Klasse zu gewährleisten.}
		\add[protected async Task InitializeConnection()]{Wird unmittelbar nach erfolgreicher Verbindung
			durch die \code{ConnectAsync()}-Methode aufgerufen, um Sensoren und ähnliches zu initialisieren.}
	\end{itemize}

\subsection{\# ISensor}
	Es erwies sich als sinnvoll, zwischen Event-basierten und asynchron auslesbaren Sensoren zu unterscheiden.
	Um einen generischen Typ zu besitzen, welcher in Klassen wie \code{List} oder \code{Dictionary} angegeben werden kann,
	existiert dieses Interface weiterhin als (ansonsten funktionslose) Überkategorie aller Sensoren.
	\begin{itemize}
		\remove[event EventHandler ValueChanged]{}
		\remove[void StartSampling()]{}
		\remove[void StopSampling()]{}
	\end{itemize}

\subsection{+ ISubscribableSensor}
	\sign{public interface ISubscribableSensor<T> : ISensor}
	Repräsentiert einen Sensor, welcher Event-basiertes Auslesen unterstützt.
	Verfügt über alle Funktionen, die zuvor durch \code{ISensor} definiert wurden.
	\begin{itemize}
		\add[event EventHandler<T> ValueChanged]{}
		\add[int SamplingRate]{Legt die Frequenz (in Hz) fest, in welcher neue Samples
			empfangen werden sollen. Unterstützt ein Sensor das Festlegen einer solchen Frequenz nicht,
			wird dies durch den Wert \code{-1} signalisiert.}
		\add[Task StartSamplingAsync()]{}
		\add[Task StopSamplingAsync()]{}
	\end{itemize}

\subsection{+ IReadableSensor}
	\sign{public interface IReadableSensor<T> : ISensor}
	Repräsentiert einen asynchron auslesbaren Sensor.
	\begin{itemize}
		\add[Task<T> ReadAsync()]{Liest asynchron den aktuellen Messwert aus.}
	\end{itemize}

\subsection{\# MotionSensor}
	\sign{public class MotionSensor : ISubscribableSensor<MotionSensorSample>, IReadableSensor<MotionSensorSample>}
	Repräsentiert die im Earable verbauten Bewegungssensoren (Gyroskop, Accelerometer).
	Sowohl \code{ISubscribableSensor}, als auch \code{IReadableSensor} werden implementiert.

\subsection{+ MotionSensorSample}
	\sign{public class MotionSensorSample : EventArgs}
	Speichert einen Momentan-Messwert des Beweungssensors.
	\begin{itemize}
		\add[public TripleShort Gyro]{Messwert des Gyroskops}
		\add[public TripleShort Acc]{Messwert des Accelerometers}
		\add[public byte SampleId]{Paketzähler, welcher mit jedem neuen Messwert hochzählt oder überläuft}
	\end{itemize}

\subsection{+ TripleShort}
	\sign{public class TripleShort}
	Struktur, welche drei 16-Bit Zahlen (\code{short}s) zusammenfasst.
	\begin{itemize}
		\add[public short x, y, z]{Die drei 16-Bit Zahlen.}
		\add[public static TripleShort FromByteArray(byte\array{} array, int offset, bool bigEndian)]{Hilfsfunktion, welche ein TripleShort
			aus einem Byte-Array ausliest.}
	\end{itemize}

\subsection{\# PushButton}
	\sign{public class PushButton : ISubscribableSensor<ButtonState>, IReadableSensor<ButtonState>}
	Repräsentiert den im Earable verbauten Zwei-Zustands-Knopf.
	Sowohl \code{ISubscribableSensor}, als auch \code{IReadableSensor} werden implementiert.

\subsection{+ ButtonState}
	\sign{public class ButtonState : EventArgs}
	Speichert einen Zustand des Zwei-Zustand-Knopfes: Gedrückt oder Nicht-Gedrückt.
	\begin{itemize}
		\add[public bool Pressed]{Der Zustand des Knopfes}
	\end{itemize}

\subsection{+ VoltageSensor}
	\sign{public class VoltageSensor : IReadableSensor<BatteryState>}
	Repräsentiert das im Earable verbaute Voltmeter, welches die am Akku anliegende Spannung misst.
	Implementiert \code{IReadableSensor}.

\subsection{+ BatteryState}
	\sign{public class ButtonState : EventArgs}
	Struktur, welche den Messwert des Voltmeters mit dem Ladezustand (Aufladen oder Entladen) zusammenfasst.
	\begin{itemize}
		\add[public float Voltage]{Die Spannung in Volt}
		\add[public bool Charging]{Der Ladezustand}
	\end{itemize}


\subsection{- IAudioStream}
	Klasse nicht mehr benötigt.

\subsection{+ ESenseMessage}
	\sign{public class ESenseMessage}
	Diese Hilfsklasse repräsentiert eine decodierte (bzw. eine noch nicht codierte) Nachricht in der Form,
	wie diese vom eSense-Earable empfangen oder zu diesem gesendet werden kann.
	\begin{itemize}
		\add[public ESenseMessage(byte header, byte\array{} data)]{Legt eine neue eSense-Nachricht mit angegebenem Kopf und Rumpf an.}
		\add[public ESenseMessage(byte\array{} received, bool hasPacketIndex = false)]{Decodiert ein Byte-Array in eine neue eSense-Nachricht.}
		\add[public byte Header]{Der Nachrichten-Kopf}
		\add[public byte\array{} Data]{Der Nachrichten-Rump}
		\add[public byte Checksum]{Aus dem Nachrichten-Rumpf berechnete Prüfsumme}
		\add[public byte? PacketIndex]{Paketzähler, welcher auch undefiniert sein kann}
		\add[public byte\array{} ToByteArray()]{(Re-)Codiert die Nachricht in ein Byte-Array}
	\end{itemize}

\subsection{+ MessageError}
	\sign{public class MessageError : Exception}
	Repräsentiert einen Fehler, welcher durch ungültige Nachrichten hervorgerufen werden kann.

\subsection{+ EarableName}
	\sign{internal class EarableName}
	Interne Hilfsklasse, welche die Funktionalitäten des Geräte-Namen-Lesens und -Schreibens bündelt.

\end{document}
