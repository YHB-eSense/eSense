\documentclass[../implementierung.tex]{subfiles}
\begin{document}

\section{StepDetection-Library}

\subsection{\# Input}
	\begin{itemize}
		\add[public int SamplingRate]{Anzahl der Datenmessungen pro Sekunde}
		\add[public Input()]{Konstruktor}
	\end{itemize}

\subsection{\# AccGyroData \changeto AccelerationSample}
	Um eine niedrigstmögliche Latzen zu erzielen, ist der Schriterkennungs-Algorithmus so implementiert, dass jeder neuen Messwert unmittelbar verarbeitet wird.
	Anstelle von mehreren Messwerten muss also nur jeweils einer an den Algorithmus übermittelt werden.
	Des weiteren ergibt sich durch die gewählte Implementierung ein Hinzukommen von Zeit-Informationen aber ein Wegfallen der Gyroskop-Daten.
	Die Struktur \code{AccGyroData} wurde dementsprechend verändert.
	\begin{itemize}
		\remove[short\array{} xacc, yacc, zacc, xgyro, ygyro, zgyro]{}
		\add[public TripleShort Acceleration]{Beschleuningungsdaten aller drei Achsen}
		\add[public DateTime Time]{Erfassungs-Zeitpunkt}
	\end{itemize}

\subsection{\# StepDetectionAlg}
	\begin{itemize}
		\add[public double IntensityOffset]{Wert, welcher im Verarbeitungsprozess des Algorithmus von der summierten Beschleuningung abgezogen wird}
		\add[public double IntensityThreshold]{Grenzwert oberhalb von welcher ein Peak als Schritt erkannt wird}
		\add[public StepDetectionAlg()]{Konstruktor für StepDetectionAlg}
	\end{itemize}

\subsection{Output}
	\begin{itemize}
		\remove[public Output(...)]{Output verwendet den Standardkonstruktor, da die Daten direkt aus dem \code{ActivityLog} des \code{SingletonOutputManager} ausgelesen werden können}
		\add[public ActivityLog Log]{}
	\end{itemize}

\subsection{\# OutputManager}
	\begin{itemize}
		\add[public ActivityLog Log]{Protokoll aller getätigten Schritte}
	\end{itemize}

\subsection{+ ActivityLog}
	Diese Klasse wurde nachträglich hinzugefügt, um fremden Komponenten eine größere Flexibilität im Zugriff auf die Schritt-Daten zu ermöglichen.
	So profitiert etwa der \code{AutoStopMode} von einer geringen Latenz, wohingegen der \code{MotivationMode} eine hohe Genauigkeit bei der berechneten Schrittfrequenz benötigt.
	Da beide Kriterien gleichzeitig nicht zu erfüllen sind, kann die Schritt-Frequenz nun auf unterschiedliche Weise berechnet werden.
	\sign{public class ActivityLog}
	\begin{itemize}
		\add[public ActivityLog(string dbPath)]{Legt einen ActivityLog mit einer Datenbank an gegebener Stelle an}
		\add[public int CountSteps(DateTime? since = null, DateTime? until = null)]{Zählt protokollierte Schritte in angegebenem Zeitraum}
		\add[public int CountSteps(TimeSpan duration)+]{Zählt protokollierte Schritte in angegebener Zeitspanne}
		\add[public List<Step> LastSteps(int n)]{Gibt einzelne Schritte zurück}
		\add[public void Add(Step step)]{Protokolliert einen neuen Schritt}
		\add[public double AverageStepFrequency(TimeSpan duration)]{Berechnet die Schritt-Frequenz über eine angegebene Zeitspanne}
		\add[public void Reset()]{Löscht das Protokoll}
	\end{itemize}

\subsection{+ Step}
	\sign{public class Step}
	\begin{itemize}
		\add[public TimeSpan Duration]{Mess-Zeitpunkt des Schrittes}
		\add[public TimeSpan Duration]{Länge des gemessenen Peaks}
		\add[public double Intensity]{Stärke des gemessenen Peaks}
	\end{itemize}

\end{document}
