\documentclass[../implementierung.tex]{subfiles}
\begin{document}
\subsection{\# eSenseSpotifyWebAPI(ursprünglich SpotifyWebAPI)}
Der Name wurde geändert, da zum Verbindungsaufbau eine Bibliothek verwendet wird in der eine Klasse mit dem Namen ' SpotifyWebAPI ' existiert.
\begin{itemize}
\add{ \textbf{SpotifyWebAPI api} Property über die Anfragen an Spotify gestellt werden }
\add{ \textbf{public PrivateProfile UsersProfile} Property für das Profil des bei Spotify angemeldeten User}
\add{ \textbf{public event EventHandler authentificationFinished} Event das ausgelöst wird wenn die Spotify Authentifizierung beendet wurde}
\add{ \textbf{public OAuth2Authenticator AuthenticationState} } Property für den Status der Authentifizierung
\add{ \textbf{public void Auth()} startet den Spotify Authentifizierungsprozess}
\end{itemize}
\subsection{\# SpotifyAudioPlayer}
\begin{itemize}
\add{ \textbf{public PrivateProfile UsersProfile} Property für das Profil des bei Spotify angemeldeten User}
\end{itemize}
\subsection{\# SpotifyAudioLib}
\begin{itemize}
\changed{\textbf{public async void TogglePause()} Ist jetzt async, das Senden des Befehls an die API lange dauert}
\changed{\textbf{public async void PlayTrack(AudioTrack track)} Ist jetzt async, das Senden des Befehls an die API lange dauert}

\end{itemize}
\subsection{+ BPMCalculator}
Berechnet die BPM zu einer .wav File, die im Konstruktor übergeben wird.
\begin{itemize}
\add{\textbf{public BPMCalculator(string file)} Speichert den Path der Datei für die spätere Berechnung der BPM}
\add{\textbf{public int Calculate()} Berechnet die BPM der vorher gewählten Datei}
\end{itemize}
\end{document}