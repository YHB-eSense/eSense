\documentclass[../implementierung.tex]{subfiles}
\begin{document}
\subsection{\# eSenseSpotifyWebAPI(ursprünglich SpotifyWebAPI)}
Der Name wurde geändert, da zum Verbindungsaufbau eine Bibliothek verwendet wird in der eine Klasse mit dem Namen ' SpotifyWebAPI ' existiert.
\begin{itemize}
\add{ Property über die Anfragen an Spotify gestellt werden. \sign{SpotifyWebAPI api}}
\add{ Property für das Profil des bei Spotify angemeldeten User. \sign{public PrivateProfile UsersProfile}}
\add{  Event das ausgelöst wird wenn die Spotify Authentifizierung beendet wurde. \sign{public event EventHandler authentificationFinished}}
\add{ Property für den Status der Authentifizierung. \sign{public OAuth2Authenticator AuthenticationState} } 
\add{  startet den Spotify Authentifizierungsprozess. \sign{public void Auth()}}
\end{itemize}
\subsection{\# SpotifyAudioPlayer}
\begin{itemize}
\add{ Property für das Profil des bei Spotify angemeldeten User.\sign{public PrivateProfile UsersProfile} }
\end{itemize}
\subsection{\# SpotifyAudioLib}
\begin{itemize}
\change{ Ist jetzt async, das Senden des Befehls an die API lange dauert. \sign{public async void TogglePause()}}
\change{ Ist jetzt async, das Senden des Befehls an die API lange dauert. \sign{public async void PlayTrack(AudioTrack track)}}

\end{itemize}
\subsection{+ BPMCalculator}
Berechnet die BPM zu einer .wav File, die im Konstruktor übergeben wird.
\begin{itemize}
\add{ Speichert den Path der Datei für die spätere Berechnung der BPM. \sign{public BPMCalculator(string file)}}
\add{ Berechnet die BPM der vorher gewählten Datei. \sign{public int Calculate()}}
\end{itemize}
\end{document}