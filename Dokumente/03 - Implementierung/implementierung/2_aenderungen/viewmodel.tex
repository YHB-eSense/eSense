\documentclass[../implementierung.tex]{subfiles}
\begin{document}
	\section{ViewModel}
		\subsection{AddSongsPageVM}
			Folgende Properties sind hinzugekommen:
			\begin{itemize}
				\add CustomColor CurrentColor bestimmt die Farbe der Buttons in der AddSongPage (readonly).
				\add string TitleLabel bestimmt welcher Text über dem Entry zum Titel angezeigt wird, dieser kann sich durch das Auswählen einer anderen Sprache ändern (readonly).
				\add string ArtistLabel bestimmt welcher Text über dem Entry zum Künstler angezeigt wird, dieser kann sich durch das Auswählen einer anderen Sprache ändern (readonly).
				\add string BPMLabel bestimmt welcher Text über dem Entry zur BPM angezeigt wird, dieser kann sich durch das Auswählen einer anderen Sprache ändern (readonly).
				\add string PickFileLabel bestimmt welchen Text der PickFileButton anzeigt, dieser kann sich durch das Auswählen einer anderen Sprache ändern (readonly).
				\add string AddSongLabel bestimmt welchen Text der AddSongButton anzeigt, dieser kann sich durch das Auswählen einer anderen Sprache ändern (readonly).
				\add ICommand GetBPMCommand ist ein Command, welcher private Methoden aufruft um die BPM einer ausgewählten Audiodatei zu berechnen (readonly).
			\end{itemize}
			Es wurden keine Properties verändert.
			Folgende Properties sind nicht mehr vorhanden:
			\begin{itemize}
				\remove string NewSongFileLocation wurde entfernt, da diese nun durch private Methoden gesetzt wird.
			\end{itemize}
			Es sind keine Methoden hinzugekommen, entfernt oder verändert wurden.
		\subsection{AudioLibPageVM}
			Folgende Properties sind hinzugekommen:
			\begin{itemize}
				\add CustomColor CurrentColor bestimmt die Farbe der ausgewählten Sortierung und der Checkboyes in der AddSongPage (readonly).
				\add string TitleLabel bestimmt welchen Text der TitleSortButton anzeigt, dieser kann sich durch das Auswählen einer anderen Sprache ändern (readonly).
				\add string ArtistLabel bestimmt welchen Text der ArtistSortButton anzeigt, dieser kann sich durch das Auswählen einer anderen Sprache ändern (readonly).
				\add string BPMLabel bestimmt welchen Text der BPMSortButton anzeigt, dieser kann sich durch das Auswählen einer anderen Sprache ändern (readonly).
				\add string PlaylistsLabel bestimmt welcher Text über dem PlaylistPicker, dieser kann sich durch das Auswählen einer anderen Sprache ändern (readonly).
				\add SimplePlaylist[] Playlists ist ein Array von Spotify-Playlists, welches in der PlaylistPicker geladen wird (readonly).
				\add SimplePlaylist SelectedPlaylist ist die momentan im PlaylistPicker ausgewählte Playlist.
				\add Color TitleSortColor ist die momentane Farbe des TitleSortButtons.
				\add Color TitleSortTextColor ist die momentane Farbe des Texts des TitleSortButtons.
				\add Color ArtistSortColor ist die momentane Farbe des ArtistSortButtons.
				\add Color ArtistSortTextColor ist die momentane Farbe des Texts des ArtistSortButtons.
				\add Color BPMSortColor ist die momentane Farbe des BPMSortButtons.
				\add Color BPMSortTextColor ist die momentane Farbe des Texts des BPMSortButtons.
				\add bool UsingBasicAudio ist ein Boolean, welcher angibt ob momentan der BasicAudioPlayer und die BasicAudioLib verwendet werden (readonly).
				\add bool UsingSpotifyAudio ist ein Boolean, welcher angibt ob momentan der SpotifyAudioPlayer und die SpotifyAudioLib verwendet werden (readonly).
				\add ICommand DeleteSongsCommand ist ein Command, welcher die ausgewählten Lieder in der AudioLibPage löscht (readonly).
				\add ICommand EditDeleteListCommand ist ein Command, welcher das ausgewählte Lied zu einer Liste von zu löschenden Lieder hinzufügt (readonly).
			\end{itemize}
			Folgende Properties wurden verändert:
			\begin{itemize}
				\changed List$<$AudioTrack$>$ Songs war zuvor eine ObservableCollection, jedoch wurden ihre speziellen Funktionalitäten nicht benötigt und deshalb zu einer einfachen Liste geändert.	
			\end{itemize}
			Es wurden keine Properties entfernt.
			Es sind keine Methoden hizugekommen.
			Folgende Methoden wurden entfernt:
			\begin{itemize}
				\remove void GetSongs() wird nicht mehr benötigt, da Songs direkt auf die Lieder im Model zugreift.	
			\end{itemize}
\end{document}
