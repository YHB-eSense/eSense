\documentclass[../implementierung.tex]{subfiles}
\begin{document}
	In diesem Dokument wird beschrieben, welche Änderungen am Entwurf von Komponente 1, der EarableLibrary, Komponente 2, der StepRecognitionLibrary und Komponente 3, der App, während der Implementierungsphase vorgenommen wurden. Weiterhin wird auf die Gründe dieser Modifizierungen eingegangen, wobei es sich dabei um veränderte, neu hinzugefügte, oder entfernte Klassen, Properties und Methoden handelt. Um möglichst einfach zwischen diesen Modifizierungen unterscheiden zu können, wurden die Symbole \#, + und - mit folgenden Bedeutungen verwendet:
	\\
	\\
	\\
	\begin{tabular}[h]{|c|l|}
	\hline 
	Symbol & Bedeutung \\
	\hline 
	+ & Klasse, Property oder Methode wurde neu hinzugefügt. \\
	\hline 
	- & Klasse, Property oder Methode wurde entfernt. \\
	\hline 
	\# & Klasse, Property oder Methode wurde verändert. \\
	\hline 
	\end{tabular}
	\\
	\\
	\\
	Einige Modifizierungen sind dadurch zu begründen, dass sich während der Implementierungen bessere oder einfachere Lösungsmöglichkeiten angeboten haben als die im Entwurf ürsprünglich spezifizierten. Ebenfalls wurden Wunschkriterien verwirklicht, die im Entwurf nicht berücksichtigt wurden weswegen vorallem im ViewModel viele Properties und im Model einige Klassen neu hinzugefügt wurden.
\end{document}
