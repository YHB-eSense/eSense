\documentclass[../implementierung.tex]{subfiles}

\begin{document}

	\section{Umgesetzte Kriterien}
		\subsection{Musskriterien}
			\begin{itemize}
				\check{Komponente 1 muss Sensordaten der Earables auslesen können}
				\check{Komponente 1 muss Audiodaten an die Earables senden können}
				\check{Komponente 2 muss - mithilfe der von Komponente 1 erlangten Sensordaten - Schritte erkennen, während der Benutzer geht oder
				läuft}
				\check{Komponente 3 muss eine Bluetooth Low-Energy Verbindung zu ausgewählten Earables aufbauen können}
				\ftsio{Komponente 3 muss eine Bluetooth Classic Verbindung zu ausgewählten Earables aufbauen können
					\indent \changeto\space Nicht praktikabel (iOS) und nicht sinvoll da OS diese Funktion übernimmt.}
				\check{Komponente 3 muss Verknüpfungen zu verschiedenen Audiodateien in einer lokalen,
				vom Benutzer bearbeitbaren Audiobibliothek speichern können}
				\check{Komponente 3 muss aus der Audiobibliothek ein zur Schrittfrequenz passendes Lied aussuchen können (Motivationsmusik-Modus)}
				\ftsio{Komponente 3 muss Audiodateien der Audiobibliothek über die Earables wiedergeben können \changeto\space Aufgabe des OS.}
				\ftsio{Komponente 3 muss ausgewählte, von Komponente 1 und Komponente 2 gewonnene Daten, anzeigen können: Gerätename und Akkuzustand
				der Earables \changeto\space Aufgabe des OS}
				\check{Alle Komponenten müssen auf Endgeräten mit iOS oder Android nutzbar sein}
				\check{Mehrsprachigkeit (Deutsch und Englisch) soll unterstützt werden}
			\end{itemize}
		\subsection{Wunschkriterien}
			\begin{itemize}
				\ftsio{Komponente 1 soll die Konfiguration der Earables verwalten können \changeto\space Nur Name kann verändert werden}
				\check{Komponente 2 soll Schritte protokollieren, um eine nachträgliche Auswertungen anzubieten (Zum Beispiel in Form von Diagrammen)}
				\check{Komponente 3 soll die Schrittanzahl anzeigen und der Benutzer diese zurücksetzen, als auch den Gerätenamen ändern können}
				\check{Komponente 3 soll einen weiteren Modus erhalten, welcher die Musik beim Stehenbleiben stoppt (AutoStop-Modus)}
				\check{Komponente 3 soll es möglich sein, den BPM-Wert einer Audiodatei automatisch (unter Zuhilfename einer externen
				Softwarebibliothek) zu ermitteln}
				\check{Es soll die Spotify API verwendet werden, um ein Lied mit passendem BPM-Wert in einer Spotify Playlist des Benutzers zu finden
				und dieses wiederzugeben}
				\check{Der Benutzer soll Komponente 3 weitere Sprachen in Form von Sprachdateien (.lang) hinzufügen können}
			\end{itemize}
		\subsection{Abgrenzungskriterien}
			\begin{itemize}
				\check{Keine der Komponenten muss auf anderen Betriebssystemen als den explizit geforderten lauffähig sein (Zum Beispiel Windows
				Phone)}
				\check{Außer ”eSense“ werden keine weiteren Earable-Plattformen unterstützt}
				\check{Komponente 3 muss von sich aus keine Musik anbieten}
				\check{Komponente 1 kann nicht das Mikrophon der Kopfhörer ansteuern}
				\check{Es kann sich nicht mit mehreren Earables auf einmal verbunden werden}
			\end{itemize}

	\section{Code-Metrik}
		Das Projekt bestitzt die folgenden Code-Metriken.
		\see{https://docs.microsoft.com/en-us/visualstudio/code-quality/code-metrics-values?view=vs-2019}
		\subsection{Werte}
		\begin{tabular}{l|l|l|l|l|l|l}
			Projekt & WI & ZK & VT & KK & ZQC & ZAC \\
			\hline
			EarableLibrary & 88 & 94 & 2 & 60 & 829 & 141 \\
			Karl & 86 & 716 & 8 & 240 & 3324 & 987 \\
			Karl.Android & 89 & 49 & 9 & 115 & 17208 & 7744 \\
			Karl.iOS & 84 & 3 & 4 & 11 & 51 & 9 \\
			StepDetectionLibrary & 86 & 71 & 1 & 35 & 583 & 92 \\
			StepDetectionTestApp & 85 & 25 & 1 & 21 & 89 & 17 \\
		\end{tabular}
		\subsection{Metrik-Tabelle}
		\begin{tabular}{l|l}
			WI & Wartbarkeitsindex \\
			ZK & Zyklomatische Komplexität \\
			VT & Vererbungstiefe \\
			KK & Klassenkopplung \\
			ZQC & Zeilen von Quellcode \\
			ZAC & Zeilen von ausführbarem Code \\
		\end{tabular}
		\\
		\\


\end{document}
