\documentclass[../entwurf.tex]{subfiles}
\begin{document}

	\subsection{Model}
		\subsubsection{Schnittstelle}
			Die Schnittstelle zu den ViewModels besteht aus Klassen, die explizit konzipiert sind um die Kommunikation zwischen ViewModels und Model 
			einfach zu gestalten. Alle Klassen der Schittstelle sind nach dem Singleton-Muster aufgebaut.
			\paragraph{public class AppLogic:}
				Diese Klasse initialisiert im Konstruktor das Model und verwaltet somit interne Abhängigkeiten im Model. Sie verwaltet gegenüber den
				ViewModels auch die Verbindung zu den Earables mithilfe der Klasse ConnectionHandler.
				\subparagraph{Attribute:}
					\begin{itemize}
						\i{private ConnectionHandler Connection} Das ConnectionHandler Objekt welches die Libraries 
						(und damit die Verbindung zu den Earables.) verwaltet.
					\end{itemize}
				\subparagraph{Properties:}
					\begin{itemize}
						\i{public AudioLib AudioLib} Das Singleton-Objekt der Schnittstellen-Klasse AudioLib. Definiert keinen Setter.
						\i{public AudioPlayer AudioPlayer} Das Singleton-Objekt der Schnittstellen-Klasse AudioPlayer. Definiert keinen Setter.
						\i{public SettingsHandler SettingsHandler} Das Singleton-Objekt der SchnittstellenKlasse SettingsHandler. 
						Definiert keinen Setter.
						\i{public ModeHandler ModeHandler} Das Singleton-Objekt der Schnittstellen-Klasse ModeHandler. Definiert keinen Setter.
						\i{public BluetoothDevice CurrentDevice} Das aktuell verbundene eSense Gerät. Definiert keinen Setter, da dieses nur vom Model
						verwaltet werden soll.
						\i{public int Steps} Die in allen Sessions seit dem letzten reset gelaufenen Schritte. Definiert keinen Setter.
						\i{public bool Connected} Boolscher Wert, ob earables verbunden sind. Definiert keinen Setter.
					\end{itemize}
				\subparagraph{Methoden:}
					\begin{itemize}
						\i{public AppLogic()} Der Konstruktor initialisiert einen ConnectionHandler.
						\i{public List<BluetoothDevice> SearchDevices()} Gibt eine Liste von gefundenen Bluetooth-Geräten zurück.
						\i{public void ConnectDevice(BluetoothDevice device)} Verbindet sich mit einem Bluetooth-Gerät.
						\i{public void DisconnectDevice()} Trennt die Verbindung zum aktuell verbundenen Bluetooth-Gerät.
					\end{itemize}
			\paragraph{class AudioLib:}
				Die Fassade für den einfachen Zugriff auf das aktuelle Audiomodul.
				Die Klasse ist desweiteren ein Singleton, da es immer genau eine Fassade gibt.
				\subparagraph{Attribute:}
					\begin{itemize}
						\i{private AudioLib \_singletonAudioLib} Dies ist das singleton Attribut.
						\i{private IAudioLibImpl AudioLibImp} Dies ist die konkrete Implementierung der
						AudioLib
					\end{itemize}
				\subparagraph{Properties:}
					\begin{itemize}
						\i{public AudioLib SingletonAudioLib} Dies ist die Property die ein Getter für den 
						Singleton definiert. Daher gibt es kein Setter.
						\i{public AudioTrack CurrentTrack} Dies ist das momentan ausgewählte Lied. Dieses
						wird auch im AudioPlayer abgepielt und angezeigt. 
						\i{public IList<AudioTrack> AudioTracks} Dies ist eine Liste aller Tracks in dieser 
						Bibliothek. Es gibt keinen Setter da diese von der konkreten Bibliothek verwaltet werden soll.
					\end{itemize}
				\subparagraph{Methoden:}
					\begin{itemize}
						\i{public void AddTrack()} Dies fügt der aktuellen konkreten Bibliothek ein neuen Track hinzu, 
						falls diese das unterstützt.
						\i{public void NextSong()} Dies überspringt den aktuellen Song und spielt einen neuen Song ab. 
						Der aktuelle als erstes Element von PlayedTracks eingefügt.
						\i{public void PrevSong()} Dies ruft \textit{CurrentTrack = AudioLibImpl.PlayedTracks.Pop()} auf.
					\end{itemize}
			\paragraph{class AudioPlayer:}
				Diese Klasse ist eine Fassade zum einfachen Zugriff auf den aktuellen konkreten AudioPlayer. 
				Sie ist nach dem Singleton Muster aufgebaut und hat die dazugehörigen Properties und Attribute, 
				da so Inkonsistenz vermieden werden kann. 
				\subparagraph{Attribute:}
					\begin{itemize}
						\i{private AudioLib AudioLib} Dies ist das Singleton Objekt von AudioLib zum einfachen Zugriff.
					\end{itemize}
				\subparagraph{Properties:}
					\begin{itemize}
						\i{public int Volume} Hier wird eine Getter und Setter für die Systemlautstärke definiert.
						\i{public double CurrentSecInTrack} Hier werden Getter und Setter für die Position im Song definiert.
						\i{public bool Paused} Hier werden Getter und Setter für den Wahrheitswert, ob das Lied pausiert ist definiert.
					\end{itemize}
				\subparagraph{Methoden:}
					\begin{itemize}
						\i{public void TogglePause()} Diese Methode negiert den Boolean Paused.
						\i{public void PlayTrack()} Diese Methode startet den Playback des CurrentSong der AudioLib
						\i{public void NextTrack()} Diese Methode ruft AudioLib.NextTrack() und PlayTrack() auf.
						\i{public void PrevTrack()} Diese Methode ruft AudioLib.PrevTrack() und PlayTrack() auf.
					\end{itemize}
		\subsubsection{Audio Modul}
			Das Audio-Modul besteht jeweils aus der Audiobibliothek und dem Audioplayer.
			Bei beiden wird zwischen der Fassade und der Implementierung unterschieden.
			Dadurch werden Konsistenzprobleme vermieden, da nicht unterschiedliche Teile der App auf unterschiedliche konkrete Implementierungen
			zugreifen können. Bzw. immer wissen, was die aktuelle Implementierung ist.
			Zusätzlich gibt es noch die Klasse AudioTrack.
			\paragraph{interface IAudioLibImpl:}
				Dieses Interface stellt eine konkrete Implementierung einer Audio Bibliothek dar.
				\subparagraph{Properties:}
					\begin{itemize}
						\i{Stack<AudioTrack> PlayedSongs} Dies ist der Stack der schon gespielten Lieder der Implementierung.
						Das erste Listenelement ist der zuletzt abgespielte Track. Es gibt keinen Setter. 
						\i{AudioTrack CurrentTrack} Dies ist der konkrete aktuelle Song.
						\i{IList<AudioTrack> AllAudioTracks} Das ist die konkrete Liste aller Songs in dieser Audio Bibliothek.
					\end{itemize}
				\subparagraph{Methoden:}
					\begin{itemize}
						\i{void AddTrack()} Dies fügt der konkreten Implementierung der Audio Bibliothek einen neuen Track hinzu, sofern
						diese das unterstützt.
					\end{itemize}
			\paragraph{interface IAudioPlayerImpl:}
				Dieses Interface stellt eine konkrete Implementierung des Audio Players dar.
				\subparagraph{Properties:}
					\begin{itemize}
						\i{double CurrentSongPos} Dies definiert Getter und Setter für die aktuelle Position im Lied.
					\end{itemize}
				\subparagraph{Methoden:}
					\begin{itemize}
						\i{void TogglePause()} Dies Pausiert den Song/Spielt den Song weiter ab.
						\i{void PlayTrack()} Dies Spielt den aktuell gewählten Song ab.
					\end{itemize}
			\paragraph{abstract class AudioTrack:}
				Diese Klasse definiert einen Abstrakten AudioTrack.
				\subparagraph{Properties:}
					Keine der Properties definiert eine set() Methode, da manche AudioLib-Implementierungen (z.B. Spotify) so etwas nicht 
					unterstützen. Neben BPM kann die Schnittstelle für weitere Eigenschaften des Songs erweitert werden.
					\begin{itemize}
						\i{public abstract double Duration} Die Dauer des Tracks.
						\i{public abstract Image Cover} Das Bild welches dem Song zugewiesen ist. Image ist eine Xamarin.Forms Klasse.
						\i{public abstract string Title} Der Name des Songs.
						\i{public abstract string Artist} Der Name des Künstlers.
						\i{public abstract int BPM} Der BPM Wert der diesem Song hinterlegt ist.
					\end{itemize}
			\paragraph{sealed class BasicAudioLib : IAudioLibImpl:}
				Diese Klasse definiert eine Standart Audio Bibliothek. Sie speichert den Ort der Audio Datei und andere Eigenschaften hinterlegter 
				Songs. Sie ist nach dem Singleton Muster aufgebaut und hat die dazugehörigen Properties und Attribute.
			\paragraph{sealed class BasicAudioPlayer : IAudioPlayerImpl}
				Diese Klasse definiert einen simplen Audio Player, welcher Audio Dateien aus dem internen Speicher auslesen und abspielen kann.
			\paragraph{sealed class BasicAudioTrack : AudioTrack}
				Diese Klasse definiert einen AudioTrack zum BasicAudioPlayer und zur BasicAudioLib.
				\subparagraph{Attribute:}
					\begin{itemize}
						\i{public String StorageLocation} Dieser String stellt dar, wo im Speicher die zugrundeliegende Audio-Datei liegt.
					\end{itemize}
			\paragraph{sealed class SpotifyAudioLib : IAudioLibImpl:}
				Diese Klasse implementiert eine konkrete Audio Bibliothek mit Spotify Integration.
				Sie hat als Grundlage eine Spotify Playlist. Sie kommuniziert mit der Klasse SpotifyWebAPI.
				Sie ist nach dem Singleton Muster aufgebaut und hat die dazugehörigen Properties und Attribute.
				\subparagraph{Attribute:}
					\begin{itemize}
						\i{private String PlaylistTag} Dieser String repräsentiert den Tag der gewählten Spotify Playlist.
					\end{itemize}
			\paragraph{sealed class SpotifyAudioPlayer : IAudioPlayerImpl:}
				Dies ist die Implementierung des Audio Player für Spotify. Die Klasse kommuniziert mit der Klasse SpotifyWebAPI.
			\paragraph{sealed class SpotifyAudioTrack : AudioTrack:}
				Diese Klasse definiert einen AudioTrack zum SpotifyAudioPlayer und zur SpotifyAudioLib.
				\subparagraph{Attribute:}
					\begin{itemize}
						\i{private String Tag} Dies repräsentiert den Tag des Spotify Track.
					\end{itemize}
			\paragraph{sealed class SpotifyWebAPI:}
				Diese Klasse dient zur Kommunikation mit der Spotify-Web-API. Sie wird die Library 
				SpotifyAPI-NET\footnote{\see[]{https://github.com/JohnnyCrazy/SpotifyAPI-NET}} verwenden. 
				Sie ist nach dem Singleton Muster aufgebaut und hat die dazugehörigen Properties und Attribute.
					

\end{document}
