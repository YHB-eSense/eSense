\documentclass[../entwurf.tex]{subfiles}
\begin{document}

	\subsection{Model}
		\subsubsection{Audio Modul}
			Das Audio-Modul besteht jeweils aus der Audiobibliothek und dem Audioplayer.
			Bei beiden wird zwischen der Fassade und der Implementierung unterschieden.
			Dadurch werden Konsistenzprobleme vermieden, da nicht unterschiedliche Teile der App auf unterschiedliche konkrete Implementierungen
			zugreifen können. Bzw. immer wissen, was die aktuelle Implementierung ist.
			Zusätzlich gibt es noch die Klasse AudioTrack.
			\paragraph{Klasse AudioLib:}
				Die Fassade für den einfachen Zugriff auf das aktuelle Audiomodul.
				Die Klasse ist desweiteren ein Singleton, da es immer genau eine Fassade gibt.
				\subparagraph{Attribute}
					\begin{itemize}
						\i{private AudioLib \_singletonAudioLib} Dies ist das singleton Attribut.
						\i{private IAudioLibImpl AudioLibImp} Dies ist die konkrete Implementierung der
						AudioLib
					\end{itemize}
				\subparagraph{Properties}
					\begin{itemize}
						\i{public AudioLib SingletonAudioLib} Dies ist die Property die ein Getter für den 
						Singleton definiert. Daher gibt es kein Setter.
						\i{public AudioTrack CurrentTrack} Dies ist das momentan ausgewählte Lied. Dieses
						wird auch im AudioPlayer abgepielt und angezeigt. 
						\i{public IList<AudioTrack> AudioTracks} Dies ist eine Liste aller Tracks in dieser 
						Bibliothek. Es gibt keinen Setter da diese von der konkreten Bibliothek verwaltet werden soll.
					\end{itemize}
				\subparagraph{Methoden}
					\begin{itemize}
						\i{public void AddTrack()} Dies fügt der aktuellen konkreten Bibliothek ein neuen Track hinzu, 
						falls diese das unterstützt.
						\i{public void NextSong()} Dies überspringt den aktuellen Song und spielt einen neuen Song ab. 
						Der aktuelle als erstes Element von PlayedTracks eingefügt.
						\i{public void PrevSong()} Dies ruft \textit{CurrentTrack = AudioLibImpl.PlayedTracks.Pop()} auf.
					\end{itemize}
			\paragraph{Interface IAudioLibImpl:}
				Dieses Interface stellt eine konkrete Implementierung einer Audio Bibliothek dar.
				\subparagraph{Properties}
					\begin{itemize}
						\i{Stack<AudioTrack> PlayedSongs} Dies ist der Stack der schon gespielten Lieder der Implementierung.
						Das erste Listenelement ist der zuletzt abgespielte Track. Es gibt keinen Setter. 
						\i{AudioTrack CurrentTrack} Dies ist der konkrete aktuelle Song.
						\i{IList<AudioTrack> AllAudioTracks} Das ist die konkrete Liste aller Songs in dieser Audio Bibliothek.
					\end{itemize}
				\subparagraph{Methoden}
					\begin{itemize}
						\i{void AddTrack()} Dies fügt der konkreten Implementierung der Audio Bibliothek einen neuen Track hinzu, sofern
						diese das unterstützt.
					\end{itemize}
			\paragraph{Class AudioPlayer:}
				Diese Klasse ist eine Fassade zum einfachen Zugriff auf den aktuellen konkreten AudioPlayer. 
				Sie ist nach dem Singleton Muster aufgebaut und hat die dazugehörigen Properties und Attribute, 
				da so Inkonsistenz vermieden werden kann. 
				\subparagraph{Attribute}
					\begin{itemize}
						\i{private AudioLib AudioLib} Dies ist das Singleton Objekt von AudioLib zum einfachen Zugriff.
					\end{itemize}
				\subparagraph{Properties}
					\begin{itemize}
						\i{public int Volume} Hier wird eine Getter und Setter für die Systemlautstärke definiert.
						\i{public double CurrentSecInTrack} Hier werden Getter und Setter für die Position im Song definiert.
						\i{public bool Paused} Hier werden Getter und Setter für den Wahrheitswert, ob das Lied pausiert ist definiert.
					\end{itemize}
				\subparagraph{Methoden}
					\begin{itemize}
						\i{public void TogglePause()} Diese Methode negiert den Boolen Paused.
						\i{public void PlayTrack()} Diese Methode startet den Playback des CurrentSong der AudioLib
						\i{public void NextTrack()} Diese Methode ruft AudioLib.NextTrack() und PlayTrack() auf.
						\i{public void PrevTrack()} Diese Methode ruft AudioLib.PrevTrack() und PlayTrack() auf.
					\end{itemize}
			\paragraph{Interface IAudioPlayerImpl:}
				Dieses Interface stellt eine konkrete Implementierung des Audio Players dar.
				\subparagraph{Properties}
					\begin{itemize}
						\i{double CurrentSongPos} Dies definiert Getter und Setter für die aktuelle Position im Lied.
					\end{itemize}
				\subparagraph{Methoden}
					\begin{itemize}
						\i{void TogglePause()} Dies Pausiert den Song/Spielt den Song weiter ab.
						\i{void PlayTrack()} Dies Spielt den aktuell gewählten Song ab.
					\end{itemize}
			\paragraph{Class SpotifyAudioLib : IAudioLibImpl:}
				Diese Klasse implementiert eine konkrete Audio Bibliothek mit Spotify Integration.
				Sie hat als Grundlage eine Spotify Playlist.
				Sie ist nach dem Singleton Muster aufgebaut und hat die dazugehörigen Properties und Attribute.
				\subparagraph{Attribute}
					\begin{itemize}
						\i{private String PlaylistTag} This String represents the tag the chosen Spotify Playlist has.
					\end{itemize}
			
					

\end{document}
