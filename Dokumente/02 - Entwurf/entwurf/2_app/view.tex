\documentclass[../entwurf.tex]{subfiles}
\begin{document}

\newcommand{\attr}[1]{\item \textit{#1}}

\subsection{View}
\subsubsection{MainPage}

\paragraph{Attribute}

\begin{itemize}
	\attr{private MainPageVM \_mainPageVM} ViewModel der MainPage
	\attr{private StackLayout.GestureRecognizers \_mainNavigationGestureRecognizers} Ist für die Navigation von der MainPage aus zuständig
	\attr{private Button \_connectToDevice} Beim Drücken startet die Suche nach verfügbaren Bluetooth Geräten
\end{itemize}

\paragraph{Methoden}
\begin{itemize}
	\attr{public MainPage(MainPageVM mainPageVM)} Initialisiert \_mainPageVM und setzt den Binding Context 			auf \_mainPageVM  
	\attr{protected override void OnAppearing()} Verhalten der Seite unmittelbar bevor sie angezeigt wird
\end{itemize}

\subsubsection{ConnectionPage}

\paragraph{Attribute}
\begin{itemize}
	\attr{private ConnectionPageVM \_connectionPageVM} ViewModel der ConnectionPage
	\attr{private ListView \_availableDevicesListView} Zeigt verfügbare eSense Earables an
	\attr{private Button \_refreshDevicesButton} Durch Drücken aktualisiert sich die Liste der verfügbaren eSense Earables
\end{itemize}

\paragraph{Methoden}
\begin{itemize}
	\attr{public ConnectionPage(ConnectionPageVM connectionPageVM)} Initialisiert \_connectionPageVM und setzt den Binding Context 			auf \_connectionPageVM
	\attr{protected override void OnAppearing()} Verhalten der Seite unmittelbar bevor sie angezeigt wird
\end{itemize}

\subsubsection{AddSongPage}

\paragraph{Attribute}
\begin{itemize}
	\attr{private AddSongPageVM \_addSongPageVM} ViewModel der AddSongPage
	\attr{private Entry \_songTitleEntry} Eingabefeld für den Namen des neuen Titel
	\attr{private Entry \_songArtistEntry} Eingabefeld für den Künstler des neuen Titel
	\attr{private Entry \_songBPMEntry} Eingabefeld für die BPM des neuen Titel
	\attr{private Button \_pickFileButton} Durch Drücken öffnet sich eine Menü in dem der User einen Song aus seinem Dateiverzeichnis auswählen kann
	\attr{private Button \_addSongButton} Durch Drücken wird der eingelesene Titel der Audiobibliothek hinzugefügt 	
\end{itemize}

\paragraph{Methoden}
\begin{itemize}
	\attr{public AddSongPage(AddSongPageVM addSongPageVM)} Initialisiert \_addSongPageVM und setzt den Binding Context 			auf \_addSongPageVM
	\attr{protected override void OnAppearing()} Verhalten der Seite unmittelbar bevor sie angezeigt wird
\end{itemize}

\subsubsection{AudioLibPage}

\paragraph{Attribute}
\begin{itemize}
	\attr{private AudioLibPageVM \_audioLibPageVM} ViewModel der AudioLibPage
	\attr{private Button \_addSongButton} Durch Drücken wird zur AddSongPage gewechselt
	\attr{private SearchBar \_searchForSongSearchBar} Bei Texteingabe der ListView aktualisiert und es werden nur noch Songs angezeigt, die das eingegebene Wort enthalten
	\attr{private ListView \_songsOfLibListView} Zeigt die verfügbaren Lieder in der Audiobibliothek an
	\attr{private Button \_sortByNameButton} Sortiert die Elemente des \_songsOfLibListView alphabetisch nach Name
	\attr{private Button \_sortByArtistButton} Sortiert die Elemente des \_songsOfLibListView nach alphabetisch nach Künstler
	\attr{private Button \_sortByBPMButton} Sortiert die Elemente des \_songsOfLibListView nach Wert der BPM (aufsteigend)
\end{itemize}

\paragraph{Methoden}
\begin{itemize}
	\attr{public AudioLibPage(AudioLibPageVM audioLibPageVM)} Initialisiert \_audioLibPageVM und setzt den Binding Context 			auf \_audioLibPageVM
	\attr{protected override void OnAppearing()} Verhalten der Seite unmittelbar bevor sie angezeigt wird
\end{itemize}

\subsubsection{AudioPlayerPage}

\paragraph{Attribute}
\begin{itemize}
	\attr{private AudioPlayerPageVM \_audioPlayerPageVM} ViewModel der AudioPlayerPage
	\attr{private Image \_songCoverImage} Zeigt das Cover des abzuspielenden Lieds an
	\attr{private Slider \_songTimeSlider} Zeigt die Position im abzuspielenden Lied an
	\attr{private Button \_previousSongButton} Durch Drücken wird das vorherige Lied abgespielt
	\attr{private Button \_previousSongButton} Durch Drücken wird das vorherige Lied abgespielt
	\attr{private Button \_playPauseSongButton} Durch Drücken wird das aktuell geladene Lied pausiert/abgespielt
	\attr{private Button \_nextSongButton} Durch Drücken wird das nächste Lied abgespielt	
	\attr{private Slider \_songVolumeSlider} Zeigt die Lautstärke des abzuspielenden Lied an
\end{itemize}

\paragraph{Methoden}

\begin{itemize}
	\attr{public AudioPlayerPage(AudioPlayerPageVM audioPlayerPageVM)} Initialisiert \_audioPlayerPageVM und setzt den Binding Context 			auf \_audioPlayerPageVM
	\attr{protected override void OnAppearing()} Verhalten der Seite unmittelbar bevor sie angezeigt wird
\end{itemize}

\subsubsection{ModesPage}

\paragraph{Attribute}
\begin{itemize}
	\attr{private ModesPageVM \_modesPageVM} ViewModel der ModesPage
	\attr{private ListView \_modesListView} Zeigt alle verfügbaren Modi an. Die hinzufügbaren ListCells bestehen aus dem Namen des Modus und einem Switch mit dem man den Modus aktivieren kann
\end{itemize}

\paragraph{Methoden}

\begin{itemize}
	\attr{public ModesPage(ModesPageVM modesPageVM)} Initialisiert \_modesPageVM und setzt den Binding Context 			auf \_modesPageVM
	\attr{protected override void OnAppearing()} Verhalten der Seite unmittelbar bevor sie angezeigt wird
\end{itemize}

\subsubsection{SettingsPage}
\paragraph{Attribute}

\begin{itemize}
	\attr{private SettingsPageVM \_settingsPageVM} ViewModel der SettingsPage
	\attr{private Entry \_deviceNameEntry} Eingabefeld für den Gerätenamen
	\attr{private Label \_selectLanguageLabel} Zeigt den Text "Select Language" (in der gewählten Sprache) an
	\attr{private Picker \_languageSelectionPicker} Picker für Sprachauswahl
	\attr{private Button \_resetStepsButton} Durch Drücken wird die auf der MainPage anzeigte Schrittzahl auf 0 zurückgesetzt
\end{itemize}

\paragraph{Methoden}

\begin{itemize}
	\attr{public SettingsPage(SettingsPageVM settingsPageVM)} Initialisiert \_settingsPageVM und setzt den Binding Context 			auf \_settingsPageVM
	\attr{protected override void OnAppearing()} Verhalten der Seite unmittelbar bevor sie angezeigt wird
\end{itemize}

\end{document}
