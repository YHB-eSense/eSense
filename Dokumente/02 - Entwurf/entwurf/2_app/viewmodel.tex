\documentclass[../entwurf.tex]{subfiles}

\begin{document}

\newcommand{\attr}[1]{\item \textit{#1}}

\subsection{ViewModel}
\subsubsection{MainPage}
\paragraph{Attribute}
\paragraph{Methoden}
\subsubsection{AddSongPage}
\paragraph{Attribute}
\begin{itemize}
	\attr{private AppLogic \_appLogic} Dieses Objekt wird genutzt um auf das Model zuzugreifen.
	\attr{public string NewSongTitle} Hier wird der vom Nutzer eingegebene Titel eines neu hinzugefügten Lieds gespeichert.
	\attr{public string NewSongArtist} Hier wird der vom Nutzer eingegebene Künstlername eines neu hinzugefügten Lieds gespeichert.
	\attr{public string NewSongBPM} Hier wird der vom Nutzer eingegebene BPM-Wert eines neu hinzugefügten Lieds gespeichert.
	\attr{public string NewSongFileLocation} Hier wird der vom Nutzer spezifizierte Pfad eines neu hinzugefügten Lieds gespeichert.
	\attr{public ICommand AddSongCommand} Dieser Command führt die private Methode \textit{AddSong()} aus.
	\attr{public ICommand PickFileCommand} Dieser Command führt die private Methode \textit{PickFile()} aus.
\end{itemize}
\paragraph{Methoden}
\begin{itemize}
	\attr{public AddSongPageVM(AppLogic appLogic)} Der Konstruktor der Klasse initialisiert die Objekte \textit{\_applogic}, \textit{AddSongCommand} und \textit{PickFileCommand}.
\end{itemize}
\subsubsection{AudioLibPage}
\begin{itemize}
	\attr{private AppLogic \_appLogic} Dieses Objekt wird genutzt um auf das Model zuzugreifen.
	\attr{private ObservableCollection<AudioTrack> \_songs} 
	\attr{public ObservableCollection<AudioTrack> Songs} 
	\attr{public ICommand TitleSortCommand} 
	\attr{public ICommand ArtistSortCommand} 
	\attr{public ICommand BPMSortCommand}
	\attr{public ICommand PlaySongCommand} 
	\attr{public ICommand AddSongCommand}
	\attr{public ICommand SearchCommand}
	\attr{public event PropertyChangedEventHandler PropertyChanged}
\end{itemize}
\paragraph{Methoden}
\begin{itemize}
	\attr{public AudioLibPageVM(AppLogic appLogic)} Der Konstruktor der Klasse initialisiert die Objekte \textit{\_applogic}, \textit{Songs} und die Commands.
	\attr{public void GetSongs()}
\end{itemize}
\subsubsection{AudioPlayerPage}
\paragraph{Attribute}
\paragraph{Methoden}
\subsubsection{ConnectionPage}
\paragraph{Attribute}
\paragraph{Methoden}
\subsubsection{MainPage}
\paragraph{Attribute}
\paragraph{Methoden}
\subsubsection{ModesPage}
\paragraph{Attribute}
\paragraph{Methoden}
\subsubsection{SettingsPage}
\paragraph{Attribute}
\paragraph{Methoden}

\end{document}
