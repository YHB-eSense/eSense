\documentclass[../entwurf.tex]{subfiles}

\begin{document}

\subsection{ViewModel}
\subsubsection{MainPage}
\paragraph{Attribute}
\paragraph{Methoden}
\subsubsection{AddSongPage}
\paragraph{Attribute}
\begin{itemize}
	\i{private AppLogic \_appLogic} Dieses Objekt wird genutzt um auf das Model zuzugreifen.
	\i{public string NewSongTitle} Hier wird der vom Nutzer eingegebene Titel eines neu hinzugefügten Lieds gespeichert.
	\i{public string NewSongArtist} Hier wird der vom Nutzer eingegebene Künstlername eines neu hinzugefügten Lieds gespeichert.
	\i{public string NewSongBPM} Hier wird der vom Nutzer eingegebene BPM-Wert eines neu hinzugefügten Lieds gespeichert.
	\i{public string NewSongFileLocation} Hier wird der vom Nutzer spezifizierte Pfad eines neu hinzugefügten Lieds gespeichert.
	\i{public ICommand AddSongCommand} Dieser Command führt die private Methode \code{AddSong()} aus.
	\i{public ICommand PickFileCommand} Dieser Command führt die private Methode \code{PickFile()} aus.
\end{itemize}
\paragraph{Methoden}
\begin{itemize}
	\i{public AddSongPageVM(AppLogic appLogic)} Der Konstruktor der Klasse initialisiert die Objekte \code{\_applogic}, \code{AddSongCommand} und \code{PickFileCommand}.
\end{itemize}
\subsubsection{AudioLibPage}
\begin{itemize}
	\i{private AppLogic \_appLogic} Dieses Objekt wird genutzt um auf das Model zuzugreifen.
	\i{private ObservableCollection<AudioTrack> \_songs} Hier werden die Songs gespeichert, die in der View angezeigt werden sollen.
	\i{public ObservableCollection<AudioTrack> Songs} Eine Property, mit der auf \code{\_songs} zugegriffen werden kann und beim setzen ein Event aufruft.
	\i{public ICommand TitleSortCommand} Dieser Command führt die private Methode \code{TitleSort()} aus.
	\i{public ICommand ArtistSortCommand} Dieser Command führt die private Methode \code{ArtistSort()} aus.
	\i{public ICommand BPMSortCommand} Dieser Command führt die private Methode \code{BPMSort()} aus.
	\i{public ICommand PlaySongCommand} Dieser Command führt die private Methode \code{PlaySong()} mit dem in der View ausgewählten Song als Parameter aus.
	\i{public ICommand AddSongCommand} Dieser Command führt die private Methode \code{AddSong()} aus.
	\i{public ICommand SearchSongCommand} Dieser Command führt die private Methode \code{SearchSong()} mit dem in der View eingegeben Text als Parameter aus.
\end{itemize}
\paragraph{Methoden}
\begin{itemize}
	\i{public AudioLibPageVM(AppLogic appLogic)} Der Konstruktor der Klasse initialisiert die Objekte \code{\_applogic}, \code{Songs} und die Commands.
	\i{public void GetSongs()} Diese Methode ruft die Songs ab, die im Model gespeichert sind und weist sie der Property \code{Songs} zu.
	\i{public event PropertyChangedEventHandler PropertyChanged} Dieses Event wird genutzt um der View mitzuteilen, dass sich eine Property geändert hat.
\end{itemize}
\subsubsection{AudioPlayerPage}
\paragraph{Attribute}
\begin{itemize}
	\i{private AppLogic \_appLogic} Dieses Objekt wird genutzt um auf das Model zuzugreifen.
	\i{private Image \_iconPlay} Hier wird das Icon des PausePlayButtons der View gespeichert, das in der View angezeigt werden soll falls der AudioPlayer pausiert ist.
	\i{private Image \_iconPause} Hier wird das Icon des PausePlayButtons der View gespeichert, das in der View angezeigt werden soll falls der AudioPlayer aktiv ist.
	\i{private AudioTrack \_audioTrack} Hier wird der momentan im Model geladene Song gespeichert.
	\i{private Boolean \_pausePlayBoolean} Dieser Boolean gibt an, ob der AudioPlayer momentan aktiv oder pausiert ist.
	\i{private int \_volume} Hier wird der Wert der Lautstärke gespeichert.
	\i{private double \_currentPosition} Hier wird die aktuelle Position im momentan aktiven Song gespeichert.
	\i{private Image \_icon} Je nachdem ob der AudioPlayer aktiv oder pausiert ist, ist diesem Image \code{\_iconPause} oder \code{\_iconPlay} zugewiesen.
	\i{public Boolean PausePlayBoolean} Eine Property, mit der auf \code{\_pausePlayBoolean} zugegriffen werden kann und beim setzen ein Event aufruft.
	\i{public AudioTrack AudioTrack} Eine Property, mit der auf \code{\_audioTrack} zugegriffen werden kann und beim setzen ein Event aufruft.
	\i{public int Volume} Eine Property, mit der auf \code{\_volume} zugegriffen werden kann und beim setzen ein Event aufruft.
	\i{public double CurrentPosition} Eine Property, mit der auf \code{\_currentPosition} zugegriffen werden kann und beim setzen ein Event aufruft.
	\i{public Image Icon} Eine Property, mit der auf \code{\_icon} zugegriffen werden kann und beim setzen ein Event aufruft.
	\i{public ICommand PausePlayCommand} Dieser Command führt die private Methode \code{PausePlay()} aus.
	\i{public ICommand PlayPrevCommand} Dieser Command führt die private Methode \code{PlayPrev()} aus.
	\i{public ICommand PlayNextCommand} Dieser Command führt die private Methode \code{PlayNext()} aus.
	\i{public ICommand ChangeVolumeCommand} Dieser Command führt die private Methode \code{ChangeVolume()} mit der in der View ausgewählten Lautstärke als Parameter aus.
	\i{public ICommand MoveInSongCommand} Dieser Command führt die private Methode \code{MoveInSong()} mit der in der View ausgewählten Position als Parameter aus.
\end{itemize}
\paragraph{Methoden}
\begin{itemize}
	\i{public AudioPlayerPageVM(AppLogic appLogic)} Der Konstruktor der Klasse initialisiert die Objekte \code{\_applogic}, die Icons und die Commands.
	\i{public void GetAudioTrack()} Diese Methode ruft den im Model momentan geladenen Song ab und weist sie der Property \code{Songs} zu.
	\i{public void GetPausePlayBoolean()} Diese Methode ruft im Model ab, ob der Audioplayer aktiv oder pausiert werden soll und weist den Boolean der Property \code{PausePlayBoolean} zu.
	\i{public void GetVolume()} Diese Methode ruft im Model ab, ob der Audioplayer aktiv oder pausiert werden soll und weist den Boolean der Property \code{PausePlayBoolean} zu.
	\i{public event PropertyChangedEventHandler PropertyChanged} Dieses Event wird genutzt um der View mitzuteilen, dass sich eine Property geändert hat.
\end{itemize}
\subsubsection{ConnectionPage}
\paragraph{Attribute}
\begin{itemize}
	\i{private AppLogic \_appLogic} Dieses Objekt wird genutzt um auf das Model zuzugreifen.
	\i{public ObservableCollection<IEarable> Devices} Hier werden die Geräte gepeichert die in der View angezeigt werden sollen.
	\i{public ICommand RefreshDevicesCommand} Dieser Command führt die private Methode \code{RefreshDevices()} aus.
	\i{public ICommand ConnectToDeviceCommand} Dieser Command führt die private Methode \code{ConnectToDevice()} mit dem in der View ausgewähltem Gerät als Parameter aus.
\end{itemize}
\paragraph{Methoden}
\begin{itemize}
	\i{public ConnectionPageVM(AppLogic appLogic)} Der Konstruktor der Klasse initialisiert die Objekte \code{\_applogic}, \code{Devices} und die Commands.
	\i{public void RefreshDevices()} Diese Methode ruft den im Model momentan erkannten Geräte ab und fügt sie der Property \code{Devices} hinzu.
	\i{public event PropertyChangedEventHandler PropertyChanged} Dieses Event wird genutzt um der View mitzuteilen, dass sich eine Property geändert hat.
\end{itemize}
\subsubsection{MainPage}
\paragraph{Attribute}
\begin{itemize}
	\i{private AppLogic \_appLogic} Dieses Objekt wird genutzt um auf das Model zuzugreifen.
	\i{private Image \_iconOn} Hier wird das Icon der View gespeichert, das in der View angezeigt werden soll falls eine Verbindung zu einem Gerät besteht.
	\i{private Image \_iconOff} Hier wird das Icon der View gespeichert, das in der View angezeigt werden soll falls keine Verbindung zu einem Gerät besteht.
	\i{private string \_deviceName} Hier wird der Name des momentan verbundenen Geräts gespeichert, falls keine Verbindung besteht, hat der String den wert null.
	\i{private string \_stepsAmount} Hier wird die Anzahl der Schritte seit dem letzten Reset gespeichert, falls keine Verbindung zu einem Gerät besteht, hat der String den wert null.
	\i{private int \_connectBoolean} Dieser Boolean gibt an, ob eine Verbindung zu einem Gerät besteht.
	\i{private Image \_icon} Je nachdem ob eine Verbindung zu einem Gerät besteht oder nicht, ist diesem Image \code{\_iconOn} oder \code{\_iconOff} zugewiesen.
	\i{public string DeviceName} Eine Property, mit der auf \code{\_deviceName} zugegriffen werden kann und beim setzen ein Event aufruft.
	\i{public string StepsAmount} Eine Property, mit der auf \code{\_stepsAmount} zugegriffen werden kann und beim setzen ein Event aufruft.
	\i{public Boolean ConnectBoolean} Eine Property, mit der auf \code{\_connectBoolean} zugegriffen werden kann und beim setzen ein Event aufruft.
	\i{public Image Icon} Eine Property, mit der auf \code{\_icon} zugegriffen werden kann und beim setzen ein Event aufruft.
	\i{public ICommand AudioPlayerPageCommand} Dieser Command führt die private Methode \code{GotoAudioPlayerPage()} aus.
	\i{public ICommand AudioLibPageCommand} Dieser Command führt die private Methode \code{GotoAudioLibPage()} aus.
	\i{public ICommand ConnectionPageCommand} Dieser Command führt die private Methode \code{GotoConnectionPage()} aus.
	\i{public ICommand ModesPageCommand} Dieser Command führt die private Methode \code{GotoModesPage()} aus.
	\i{public ICommand SettingsPageCommand} Dieser Command führt die private Methode \code{GotoSettingsPage()} aus.
\end{itemize}
\paragraph{Methoden}
\begin{itemize}
	\i{public MainPageVM(AppLogic appLogic)} Der Konstruktor der Klasse initialisiert die Objekte \code{\_applogic}, die Icons und die Commands.
	\i{public void GetDeviceName()} Diese Methode ruft den Namen des momentan verbundenen Gerätes ab und weist sie der Property \code{DeviceName} zu.
	\i{public void GetStepsAmount()} Diese Methode ruft im Model ab, wie viele Schritte seit dem letztem Reset gemacht wurden und weist den Wert der Property \code{StepsAmount} zu.
	\i{public void GetConnectBoolean()} Diese Methode ruft im Model ab, ob momentan eine Verbindung zu einem Gerät besteht und weist den Boolean der Property \code{ConnectBoolean} zu.
	\i{public event PropertyChangedEventHandler PropertyChanged} Dieses Event wird genutzt um der View mitzuteilen, dass sich eine Property geändert hat.
\end{itemize}
\subsubsection{ModesPage}
\paragraph{Attribute}
\begin{itemize}
	\i{private AppLogic \_appLogic} Dieses Objekt wird genutzt um auf das Model zuzugreifen.
	\i{private ObservableCollection<Mode> \_modes} Hier werden die Modi gespeichert, die in der View angezeigt werden sollen.
	\i{public ObservableCollection<Mode> Modes} Eine Property, mit der auf \code{\_modes} zugegriffen werden kann und beim setzen ein Event aufruft.
	\i{public ICommand ActivateModeCommand} Dieser Command führt die private Methode \code{ActivateMode()} mit dem in der View ausgewählten Modus als Parameter aus.
\end{itemize}
\paragraph{Methoden}
\begin{itemize}
	\i{public ModesPageVM(AppLogic appLogic)} Der Konstruktor der Klasse initialisiert die Objekte \code{\_applogic}, \code{Modes} und \code{ActivateModeCommand}.
	\i{public void GetModes()} Diese Methode ruft die im Model vorhandenen Modi ab und weist sie der Property \code{Modes} zu.
	\i{public event PropertyChangedEventHandler PropertyChanged} Dieses Event wird genutzt um der View mitzuteilen, dass sich eine Property geändert hat.
\end{itemize}
\subsubsection{SettingsPage}
\paragraph{Attribute}
\begin{itemize}
	\i{private AppLogic \_appLogic} Dieses Objekt wird genutzt um auf das Model zuzugreifen.
	\i{private ObservableCollection<Lang> \_languages} Hier werden die Sprachen gespeichert, die in der View angezeigt werden sollen.
	\i{private Lang \_selectedLanguage} Hier wird die Sprache gespeichert, die in der View als ausgewählt angezeigt wird.
	\i{private string \_deviceName} Hier wird der Name des momentan verbundenen Geräts gepeichert der in der View angezeigt wird.
	\i{public ObservableCollection<Lang> Languages} Eine Property, mit der auf \code{\_languages} zugegriffen werden kann und beim Setzen ein Event aufruft.
	\i{public Lang SelectedLanguage} Eine Property, mit der auf \code{\_selectedLanguage} zugegriffen werden kann und beim Setzen ein Event aufruft.
	\i{public string DeviceName} Eine Property, mit der auf \code{\_deviceName} zugegriffen werden kann und beim Setzen ein Event aufruft.
	\i{public ICommand ChangeDeviceNameCommand} Dieser Command führt die private Methode \code{ChangeDeviceName()} mit dem in der View eingegebenen Text als Parameter aus.
	\i{public ICommand ChangeLanguageCommand} Dieser Command führt die private Methode \code{ChangeLanguage()} mit der in der View ausgewählten Sprache als Parameter aus.
	\i{public ICommand ResetStepsCommand} Dieser Command führt die private Methode \code{ResetSteps()} aus.
\end{itemize}
\paragraph{Methoden}
\begin{itemize}
	\i{public SettingsPageVM(AppLogic appLogic)} Der Konstruktor der Klasse initialisiert die Objekte \code{\_applogic} und die Commands.
	\i{public void RefreshLanguages()} Diese Methode ruft die im Model vorhandenen Sprachen ab und weist sie der Property \code{Languages} zu.
	\i{public void GetSelectedLanguage()} Diese Methode ruft die im Model ausgewählte Sprache ab und weist sie der Property \code{SelectedLanguage} zu.
	\i{public void GetDeviceName()} Diese Methode ruft den Namen des momentan verbundenen Gerätes im Model ab und weist ihn der Property \code{DeviceName} zu.
	\i{public event PropertyChangedEventHandler PropertyChanged} Dieses Event wird genutzt um der View mitzuteilen, dass sich eine Property geändert hat.
\end{itemize}
\subsubsection{NavigationHandler}
\paragraph{Attribute}
\begin{itemize}
	\i{static private AudioPlayerPage \_audioPlayerPage}
	\i{static private AudioLibPage \_audioLibPage}
	\i{static private ConnectionPage \_connectionPage}
	\i{static private ModesPage \_modesPage}
	\i{static private SettingsPage \_settingsPage}
	\i{static private AddSongPage \_addSongPage}
	\i{static private MainPage \_mainPage}
\end{itemize}
\paragraph{Methoden}
\begin{itemize}
	\i{public SettingsPageVM(AppLogic appLogic)} Der Konstruktor der Klasse initialisiert alle Attribute.
	\i{public static async void GotoAudioPlayerPage()} Diese Methode ändert die Ansicht der View auf die \code{\_audioPlayerPage}.
	\i{public static async void GotoAudioLibPage()} Diese Methode ändert die Ansicht der View auf die \code{\_audioLibPage}.
	\i{public static async void GotoConnectionPage()} Diese Methode ändert die Ansicht der View auf die \code{\_connectionPage}.
	\i{public static async void GotoModesPage()} Diese Methode ändert die Ansicht der View auf die \code{\_modesPage}.
	\i{public static async void GotoSettingsPage()} Diese Methode ändert die Ansicht der View auf die \code{\_settingsPage}.
	\i{public static async void GotoAddSongPage()} Diese Methode ändert die Ansicht der View auf die \code{\_addSongPage}.
	\i{public static async void GotoMainPage()} Diese Methode ändert die Ansicht der View auf die \code{\_mainPage}.
	\i{public static async void GoBack()} Diese Methode ändert die Ansicht der View auf die Page die zuvor angezeigt wurde, jedoch nur falls die Ansicht nicht \code{\_mainPage} entspricht.
\end{itemize}

\end{document}
