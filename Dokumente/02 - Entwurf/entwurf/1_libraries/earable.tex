\documentclass[../entwurf.tex]{subfiles}
\begin{document}

\section{Earable-Library}

\subsection{ConnectionHandler}
Stellt Interaktionsmöglichkeiten mit der unterliegenden Bluetooth-Schnittstelle bereit.
\paragraph{Events}
\begin{itemize}
	\i{public event EventHandler EarableDiscovered} Erteilt jedes Mal eine Benachrichtigung, wenn bei aktiver Suche ein kompatibles Bluetooth-Gerät gefunden wurde.
\end{itemize}
\paragraph{Methoden}
\begin{itemize}
	\i{public void StartScanning()} Startet die Suche nach Bluetooth-Geräten, falls nicht bereits aktiv.
	\i{public void StopScanning()} Beendet die Suche nach Bluetooth-Geräten, falls aktiv.
\end{itemize}

\subsection{IEarable}
Allgemeine Schnittstelle für Earable-Geräte.
\paragraph{Attribute}
\begin{itemize}
	\i{string Name} Ermöglicht zugriff auf den vergebenen Geräte-Namen.
	\i{Guid Id} Die Bluetooth-Adresse des Geräts.\glsnote{ro}
	\i{IAudioStream AudioStream} Zugehöriges AudioStream-Objekt.\glsnote{ro}
	\i{ReadOnlyCollection\footnote{System.Collections.ObjectModel}<ISensor> sensors} Liste aller verfügbaren Sensoren.\glsnote{ro}
\end{itemize}
\paragraph{Methoden}
\begin{itemize}
	\i{Task\footnote{System.Threading.Tasks.Task}<bool> ConnectAsync()} Versucht asynchron eine Verbindung aufzubauen, sofern diese nicht bereits besteht.
	\i{Task<bool> DisconnectAsync()} Trennt die aktuelle Verbindung asynchron. Gibt \code{false} zurück, falls keine Verbindung besteht.
	\i{bool IsConnected()} Gibt einen Wahrheitswert zurück, der angibt, ob aktuell eine Verbindung besteht.
\end{itemize}

\subsection{ISensor}
Allgemeine Schnittstelle für Sensoren.
\paragraph{Events}
\begin{itemize}
	\i{event EventHandler ValueChanged} Erteilt Benachrichtigungen im Falle einer Messwertaktualisierung.
\end{itemize}
\paragraph{Mögliche Implementierungen}
\begin{itemize}
	\i{MotionSensor} Repräsentiert die im Earable verbauten Bewegungssensoren (Gyroskop, Accelerometer)
	\i{PushButton} Repräsentiert den im Earable verbauten Zwei-Zustands-Knopf.
\end{itemize}

\subsection{IAudioStream}
Schnittstelle für einen Audio-Ausgabe-Stream.
\paragraph{Methoden}
\begin{itemize}
	\i{bool open()} Öffnet den Stream.
	\i{bool close()} Schließt den Stream.
	\i{void write(byte[] data)} Hängt Audiodaten an den Buffer des AudioStreams an.
\end{itemize}

\end{document}
