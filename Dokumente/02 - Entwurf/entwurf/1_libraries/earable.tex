\documentclass[../entwurf.tex]{subfiles}
\begin{document}

\newcommand{\code}[1]{\texttt{#1}}
\renewcommand{\i}[1]{\item \code{#1}}

\subsection{Earable-Library}

\subsubsection{ConnectionHandler}
Stellt Interaktionsmöglichkeiten mit der unterliegenden Bluetooth-Schnittstelle bereit.
\paragraph{Attribute}
\begin{itemize}
	\i{public event EventHandler EarableDiscovered} Erteilt jedes Mal eine Benachrichtigung, wenn bei aktiver Suche ein kompatibles Bluetooth-Gerät gefunden wurde.
\end{itemize}
\paragraph{Methoden}
\begin{itemize}
	\i{public void StartScanning()} Startet die Suche nach Bluetooth-Geräten, falls nicht bereits aktiv.
	\i{public void StopScanning()} Beendet die Suche nach Bluetooth-Geräten, falls aktiv.
\end{itemize}


\subsubsection{IEarable}
Interface, welches ein Earable-Gerät repräsentiert.
\paragraph{Attribute}
\begin{itemize}
	\i{string Name} Ermöglicht zugriff auf den vergebenen Geräte-Namen.
	\i{Guid Id} Die Bluetooth-Adresse des Geräts.\footnotemark[1]
	\i{IAudioStream AudioStream} Zugehöriges AudioStream-Objekt.\footnotemark[1]
	\i{ReadOnlyCollection$<$ISensor$>$ sensors} Liste aller verfügbaren Sensoren.\footnotemark[1]
\end{itemize}
\paragraph{Methoden}
\begin{itemize}
	\i{System.Threading.Tasks.Task<bool> ConnectAsync()} Versucht asynchron eine Verbindung aufzubauen, sofern diese nicht bereits besteht.
	\i{System.Threading.Tasks.Task<bool> DisconnectAsync()} Trennt die aktuelle Verbindung asynchron. Gibt \code{false} zurück, falls keine Verbindung besteht.
	\i{bool IsConnected()} Gibt einen Wahrheitswert zurück, der angibt, ob aktuell eine Verbindung besteht.
\end{itemize}
\footnotetext[1]{Nur Lesezugriff}
\end{document}
