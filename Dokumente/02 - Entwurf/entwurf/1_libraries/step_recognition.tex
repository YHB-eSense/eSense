\documentclass[../entwurf.tex]{subfiles}
\begin{document}

\section{StepDetection-Library}
    \subsection{Input}
        \sign{public class Input : IObservable<AccGyroData>}
        Nimmt Beschleuninigunsdaten entgegen und gibt diese an Beobachter als struct AccGyroData weiter.
        \paragraph{Methoden}
        \begin{itemize}
            \i{public IDisposable Subscribe(IObserver<AccGyroData> observer)} Fügt neue Beobachter hinzu.
            \i{public void Update(AccGyroData data)} Pusht Daten zu den Beobachtern.
            \i{public void ValueChanged(object sender, EventArgs args)} Nimmt Beschleuninigunsdaten entgegen.
        \end{itemize}


    \subsection{AccGyroData}
        \sign{public struct AccGyroData}
        Enthält Beschleuninigunsdaten.
        \paragraph{Properties:}
        \begin{itemize}
            \i{public short[] xacc} Beschleuninigunsdaten x-Achse.
            \i{public short[] yacc} Beschleuninigunsdaten y-Achse.
            \i{public short[] zacc} Beschleuninigunsdaten z-Achse.
            \i{public short[] xgyro} Beschleuninigunsdaten x-Drehung.
            \i{public short[] ygyro} Beschleuninigunsdaten y-Drehung.
            \i{public short[] zgyro} Beschleuninigunsdaten z-Drehung.
        \end{itemize}

    \subsection{StepDetectionAlg}
        \sign{class StepDetectionAlg :IObserver<AccGyroData>, IObservable<Output>}
        Erhält struct AccGyroData von Input und berechnet anhand der Daten Schrittfrequenz und Anzahl der Schritte
        \paragraph{Properties:}
        \begin{itemize}
            \i{private List<IObserver<Output>> observers} Liste mit Beobachtern.
        \end{itemize}
    
        \paragraph{Methoden}
        \begin{itemize}
            \i{public void OnCompleted()} Wenn Subjekt fertig mit der Datenübertragung ist.
            \i{public void OnError(Exception error)} Wenn Subject einen Fehler erfährt.
            \i{public void OnNext(AccGyroData value)} Enthält Daten von Subjekt.
            \i{public IDisposable Subscribe(IObserver<Output> observer)} Fügt neue Beobachter hinzu.
            \i{public void Update(Output output)} Pusht Daten zu Beobachtern.
            \i{private void StepDetecAlg(AccGyroData data)} Berechnet Schrittfrequenz und Anzahl der Schritte.
        \end{itemize}
    
    \subsection{Outputmanager}
        \sign{public class OutputManager : IObservable<Output>, IObserver<Output>}
        Gibt die berechneten Schrittdaten an Anwender weiter.
        \paragraph{Properties:}
        \begin{itemize}
            \i{private static OutputManager singletonOutputManager} Enthält ein OutputManager-Objekt um das Singleton Entwurfsmuster zu implementieren.
            \i{private List<IObserver<Output>> observers} Liste der Beobachter.
            \i{public static OutputManager SingletonOutputManager} \glsnote{ro}
        \end{itemize}
        \paragraph{Methoden}
        \begin{itemize}
            \i{public void OnCompleted()} Wenn Subjekt fertig mit der Datenübertragung ist.
            \i{public void OnError(Exception error)} Wenn Subject einen Fehler erfährt.
            \i{public void OnNext(AccGyroData value)} Enthält Daten von Subjekt.
            \i{public IDisposable Subscribe(IObserver<Output> observer)} Fügt neue Beobachter hinzu.
            \i{public void Update(Output output)} Pusht Daten zu Beobachtern.
        \end{itemize}
    

    \subsection{Output}
        \sign{public struct Output}
        \paragraph{Properties:}
        \begin{itemize}
            \i{private double freq} Schrittfrequenz.
            \i{private int stepcount} Anzahl der Schritte.
            \i{public double Frequency} \glsnote{ro}
            \i{public int StepCount} \glsnote{ro}
        \end{itemize}
        \paragraph{Methoden}
        \begin{itemize}
            \i{public Output(double freq, int stepcount)} Konstruktor für Output.
        \end{itemize}


                        
\end{document}
