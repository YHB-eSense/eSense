
	\chapter{Zielbestimmung}
		Software-Entwickler sollen in die Lage versetzt werden mit Hilfe von Produkt 1, eine Activity Recognition Library, Sensordaten der Earables auszulesen. 
		Weiterhin soll es Softwareentwicklern möglich gemacht werden mit Hilfe von Produkt 2, einem Schritterkennungsmodul, die Schrittfrequenz zu erkennen.
		Produkt 3, eine App für iOS und Android, soll einen konkreten Anwendungsfall von Produkt 1 und 2 darstellen.
		\section{Musskriterien}
		\begin{itemize}
			\item Produkt 1 muss in der Lage sein Sensordaten der Earables auszulesen.
			\item Produkt 1 muss in der Lage sein, den Audio-Output der Earables zu steuern.
			\item (Produkt 2 muss in der Lage sein mithilfe der von Produkt 1 erlangten Sensordaten Schritte zu erkennen.) ?
			\item Produkt 2 muss mithilfe der Schritterkennung Schrittmuster erkennen.
			\item Produkt 3 muss ein oder mehrere konkrete Anwendungsfälle von Produkt 1 und 2 implementieren.
		\end{itemize}
		\section{Wunschkriterien}
		\begin{itemize}
			\item Produkt 1 soll in der Lage sein die Konfiguration der Earables zu verwalten.
			\item Produkt 2 soll Auswertungen der Schrittmuster anbieten. (Zum Beispiel Schrittfrequenz.)
			\item Produkt 3 soll die von Produkt 2 angebotenen Auswertungen anzeigen können.
		\end{itemize}
		\section{Abgrenzungskriterien}